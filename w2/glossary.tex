
\newglossaryentry{caricatura}{name=Caricatura,description={Las caricaturas son una forma de arte que consiste en realizar una representación exagerada y humorística de una persona o situación. Se utilizan técnicas de dibujo para resaltar ciertas características físicas o de personalidad y se suelen utilizar colores vivos y formas exageradas.}}

\newglossaryentry{comico}{name=Cómico,description={El término "cómico" se refiere a alguien o algo que hace reír o provoca risa. Puede referirse a un humorista, comediante o actor que se dedica a hacer reír a la gente a través de su actuación o sus chistes. También puede referirse a una situación o historia divertida que tiene como objetivo principal provocar risa en el público. En general, un cómico es alguien que se caracteriza por su habilidad para generar humor y alegría en los demás.}}

\newglossaryentry{esbozo}{name=Esbozo,description={En español, la palabra "esbozo" se refiere a la acción de crear una representación inicial o preliminar de algo, ya sea una idea, un dibujo o un plan. Un esbozo puede ser el punto de partida para desarrollar una obra completa, como un boceto antes de hacer un dibujo detallado o un bosquejo antes de escribir un ensayo. También se puede utilizar en sentido figurado para referirse a una descripción general o resumida de algo.}}

\newglossaryentry{periodismo}{name=Periodismo,description={El periodismo es una forma de comunicación que consiste en recopilar, verificar y difundir información sobre eventos, acontecimientos y temas de interés público. Los periodistas son responsables de investigar, escribir y presentar noticias de manera objetiva y precisa.}}

\newglossaryentry{social}{name=Social,description={El término "social" puede tener diferentes significados dependiendo del contexto en el que se utilice. En general, se refiere a todo lo relacionado con la sociedad, las relaciones entre las personas y la interacción social.}}

\newglossaryentry{interrelacion}{name=Interrelación,description={La interrelación se refiere a la conexión o relación que existe entre diferentes elementos, fenómenos o individuos. Implica que dos o más cosas están conectadas, afectándose mutuamente o influyéndose entre sí. La interrelación puede ser de diversos tipos y puede darse en distintos ámbitos, como el social, económico, político, natural, etc.}}

\newglossaryentry{conceptos}{name=Conceptos,description={Una unidad cognitiva de significado. Nace como una idea abstracta (es una construcción mental) que permite comprender las experiencias surgidas a partir de la interacción con el entorno y que, finalmente, se verbaliza (se pone en palabras).}}



\newacronym{ESFAPA}{ESFAPA}{Escuela superior de formación artística pública de Ayacucho}
%\newacronym{IFS}{IFS}{Iterated function system (sitemas de funciones iteradas)}

%\newacronym{VI}{V1}{Variable independiente}
%\newacronym{VD}{V2}{Variable dependiente}
%\newacronym{V}{V}{Variable}

%\newacronym{D}{D}{Dimensiones}

\newacronym{M}{M}{Malo}
\newacronym{B}{B}{Bueno}
\newacronym{R}{R}{Regular}
\newacronym{E}{E}{Excelente}
\newacronym{O}{O}{Observación}

\newacronym{IT}{IT}{$i$--ésimo items de los indicadores de la variable independiente}
\newacronym{ID}{ID}{Identificador de numeración}
\newacronym{Ci}{Ci}{$i$--ésimo items de la variable independiente}
\newacronym{Pi}{Pi}{$i$--ésimo items de la variable dependiente}

%%%%%%%%%%%%%%%%%%%%%%%%%%

%\glsxtrnewsymbol[description={Promedio de las observaciones correspondientes a la dimencion $i$ de la variable dependiente, donde $j=1,2,3,4,5$, $i=1,2,3$ son los promedios en cada observacion $O^i$ de las dimension $Di$ con el método tradicional y experimental respectivamente}]{w1}{$O^i_{\overline{x}_{Dj}^T}$ y $O^i_{\overline{x}_{Dj}^E}$}

\glsxtrnewsymbol[description={Sumatoria de las varianzas de cada elemento muestral}]{var}{$\sum s^2_i$}

\glsxtrnewsymbol[description={Varianzas de cada item}]{var2}{$s^2_i$}

%\glsxtrnewsymbol[description={Promedio total}]{pi1}{$\overline{x}_T$}

%\glsxtrnewsymbol[description={Calificaciones de la variable dependiente en la experimentación tradicional y experimental respectivamente}]{pi2}{$\overline{V}^T$ y $\overline{V}^E$}

%\glsxtrnewsymbol[description={Frecuencia absoluta simple}]{f}{$f$}

%\glsxtrnewsymbol[description={Promedio de los items correspodientes a la dimensión de la variable dependiente con el método tradicional y experimental respectivamente}]{xD1}{$\overline{x}_{Di}^T$ y $\overline{x}_{Di}^E$}

%\glsxtrnewsymbol[description={Promedio de los items correspodientes a la variable dependiente	con el método tradicional y experimental respectivamente}]{xD2}{$\overline{x}_{T}^T$ y $\overline{x}_{T}^E$}