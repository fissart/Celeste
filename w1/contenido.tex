\usepackage[apaciteclassic, numberedbib, nosectionbib, tocbib]{apacite}
\usepackage{usebib}
\bibinput{une}
%<*agradecimiento>
A la \lugar, a la conjunto de docentes, quienes en el transcurso de mis estudios.

durante los años de estudio supieron guiar mi formación profesional impartiendo sus conocimientos
artísticos.

Al docente \asesor, asesor de la esta investigación por su colaboración en la estructuración y elaboración 

A los docentes: \asesor, \expert y \expertt, quienes verificaron
y validaron los instrumentos de recolección de datos y las sugerencias
proporcionadas.

A los estudiantes y docentes de la \lugar, quienes posibilitaron la recolección de datos del presente trabajo de investigación.

A las personas y amistades que de algún modo colaboraron en la realización
de esta investigación.
%</agradecimiento>






%<*intro>
La investigación se realizó debido a que, en la \lugar, donde tradicionalmente
los docentes tienen poca creatividad, en cuanto al uso de los principios
de la composición plástica tridimensional en base a la geometría fractal,
implicando en el bajo rendimiento \emph{compositivo de los estudiantes.}
Se realiza esta investigación debido a la necesidad primordial de
mejorar, la metodología de los docentes, en la enseñanza de la composición
y colaborar con un modelo estructural didáctico más, entre muchos.
La característica abstracta de las geometría fractal no la excusa,
de hacerla puramente simbólica, es más esta es multidisciplinaria,
en la que participan diversos factores, entre las que se destaca el
aspecto visual de los objetos tridimensionales, con esta analogía
es posible descubrir más de lo evidente en las artes. Por tanto que
este aspecto colabore en la adquisición cognitiva de la composición
plástica tridimensional. En la enseñanza tradicional se obvia en demasía
la geometría fractal debido a las dificultades o falencias de los
docentes, probablemente por el bajo interes, de esta característica
en la enseñanza.

En las distintas carreras de la universidad se estudian teorías considerando
la representación del objeto estudiado, con el objetivo de descubrir
aspectos que las palabras o símbolos, no las pueden describir, lo
que implica considerar importante la representación y la estética
de los objetos tridimensionales, en la enseñanza aprendizaje de los
estudiantes universitarios, haciendo más ameno este proceso en su
vida educativa, favoreciendo el alcance de una actitud crítica de
lo estudiantes.

En resumen esta investigación se realizó porque se tuvo la necesidad
de proponer estrategias, en la solución de problemas de la composición
plástica tridimensional en el campo de las artes plásticas y generar
razones acerca de la utilidad y aplicabilidad de los resultados del
estudio. En particular \MakeTextLowercase{\objetivo}.

Resulta oportuno modificar nuestros métodos tradicionales de enseñanza
artística universitaria con fines de Acreditación y Certificación
de la Calidad Universitaria en la que estamos involucrados docentes,
estudiantes y comunidad universitaria los que nos exigen productos
de calidad, competentes, altamente eficientes y creativos.

Las razones expuestas, motivaron la realización del presente trabajo
investigación, titulada \MakeTextLowercase{\titulo}; donde las variables
de estudio son, la \MakeTextLowercase{\variablei} y el \MakeTextLowercase{\variabled},
los cuales se analizaron con el objetivo de lograr evaluar, la repercusión
de la primera en la segunda; se formuló el objetivo, \MakeTextLowercase{\objetivo},
a fin de contribuir en el campo del conocimiento artístico plástico
y la práctica constructiva, para mejorar los principios de la composición
plástica tridimensional: \MakeTextLowercase{\dimd}, \MakeTextLowercase{\dimdd},
\MakeTextLowercase{\dimddd}.
El fundamento teórico del presente trabajo de investigación, está
enmarcado con el enfoque, composición plástica para mejorar las competencias
plásticas compositivas. El contenido del presente trabajo de investigación
está estructurado en cuatro capítulos, en el primer capítulo trata
acerca del planteamiento del problema, el segundo se refiere al marco
teórico, el tercer capítulo concierne a la metodología de investigación,
el cuarto capítulo referido a los resultados de la investigación con
la discusión respectiva, finalmente conclusiones y sugerencias
%</intro>

%<*resumen>

El presente trabajo de investigación tuvo como objetivo \MakeTextLowercase{\objetivo}.
Tipo de investigación básica, nivel de investigación experimental
explicativa, de diseño cuasiexperimental de un grupo con pre y posprueba
en series temporales equivalentes (alternado); se empleó el método
hipotético deductivo, experimental y estadístico descriptivo e inferencial;
el lugar de estudio fue en la \lugar; la muestra fue no probabilística
e intencional compuesta por un solo grupo experimental de 35 estudiantes
de la serie 300, matriculados en el curso de arquitectura, del semestre
impar de la escuela profesional de Ingeniería Civil UNSCH; los datos
fueron recolectados a través de la prueba escrita y la ficha de observación;
la prueba de validez de instrumentos se realizó a través de juicio
de expertos y la confiabilidad, a través de prueba del Coeficiente
de Pearson y la corrección de Spearman Brow. Se verificó la no normalidad
de los datos, mediante la prueba de \emph{Shapiro Wilks}; se
aplicó la prueba de Student dos muestras relacionadas para la prueba
de hipótesis, con un nivel de confianza del 95\% y significancia del
5\% y se concluyó, que \MakeTextLowercase{\hipotesis}.

Palabras Claves: \variablei, \MakeTextLowercase{\variabled}, \dimi, \dimii, \dimd, \dimdd, \dimddd.
%</resumen>

%<*abstrac>
The objective of this investigation was to determine the influences
of the application of the fractal geometry in the plastic composition
of the students of The School of Professional Training of Civil Engineering
UNSCH 2019. Type of research. In the basics, level of explanatory
experimental research, of non-quasi-experimental design a group with
previous and subsequent tests in equivalent time series (alternate);
the job was hypothetical deductive, descriptive and descriptive and
inferential statistical method; the place of the study was carried
out in the School of Professional Training of Civil Engineering UNSCH
- Ayacucho; The sample was non-probabilistic and intentional composed
of a single experimental group of 35 students from the 300 series,
enrolled in the architecture course, from the odd semester. from the
UNSCH civil engineering professional school; The data was collected
through of the written test and the observation form; The validity
test of the instruments is performed. through expert judgment and
reliability, through the Pearson coefficient test and the correction
of Spearman's forehead. The non-normality of the data is verified,
through the Test of Shapiro Wilks; Two related samples are applied
to the student's test. for the hypothesis test, with a confidence
level of 95\% and a significance of 5\% and He concluded that the
application in fractal geometry significantly influences development.
Of the plastic composition of the students of the Training School.
Professional in Civil Engineering UNSCH 2019. 

Keywords: Fractal geometry, three-dimensional plastic composition, fractals
natural, abstract fractals, variety, unity, rhythm, balance and emphasis.
%</abstrac>



%<*importancia>
El presente proyecto queda justificado  en el \lugar, por los siguientes ítems:
%</importancia>

%<*justificacionconcepto>
El presente proyecto queda justificado  en el \lugar, por los siguientes ítems:
%</justificacionconcepto>

%<*justificacionteo>
El principal objetivo de la  investigación es aportar al conocimiento existente en las didácticas matemáticas, esto es, sobre la relación de la \MakeTextLowercase{\variablei}, con las \MakeTextLowercase{\variabled}, cuyos resultados se podrá sistematizarse en una propuesta para ser incorporado como conocimiento a las ciencias de la educación matemática, ya que se estaría demostrando que existe relación directa y significativa entre estas dos variables de estudio.

%</justificacionteo>










%<*justificacionped>
La aplicación de la \MakeTextLowercase{\variablei} en la didáctica en el desarrollo de las capacidades o competencias matemáticas se indaga mediante métodos científicos, situaciones que pueden ser investigadas por la ciencia, una vez que sean demostrados su validez y confiabilidad se podrán utilizar en otros trabajos de investigación y en diversas instituciones educativas.
%</justificacionped>


%<*justificacionsoc>
La investigación permitirá dar mayor importancia al desarrollo de la 
\MakeTextLowercase{\variablei} en los estudiantes de la \lugar y otras
instituciones similares en la región y el país. Asimismo será de utilidad para el
ejercicio de nuestro trabajo como docentes en secundaria y superior,
para orientar y brindar servicios de tutoría a las necesidades que presenten los estudiantes en el plano académico, social, y productivo mediante la prueba de la hipotesis: \MakeTextLowercase{\hipotesis}.
%</justificacionsoc>





\subsection{Enfoque de la investigación}
El presente trabajo de investigación se fundamenta en el enfoque cuantitativo, porque, utiliza la recolección de datos numéricos para probar hipótesis con base en la medición numérica y el análisis estadístico descriptivo e inferencial, con el fin establecer pautas de comportamiento y probar teorías.

De acuerdo a \cite{bernal} La orientación cuantitativa se fundamenta en el cálculo de características de los fenómenos sociales, lo cual presupone derivar de un cuadro conceptual adecuado al problema examinado, una cadena de proposiciones que enuncien relaciones entre si las variables experimentadas de forma justificada. Este procedimiento tiende a extender y sistematizar resultados. Según \cite{inv1}.

\begin{displayquote}
{La orientación cuantitativa se constituye, de un conjunto de términos que es secuencial y demostrativo. Cada fase antecede a la subsiguiente y no es posible omitir o evitar sucesos. La disposición es inflexible. Parte de una idea que va delimitándose y, una vez definida, se proceden a generar objetivos y cuestiones de exploración, se revisa la bibliografía y se elabora un marco o una configuración teórica. De las cuestiones se forman hipótesis y establecen variables; se bosqueja un procedimiento para experimentar la cual se conoce como diseño, se calculan las variables en un determinado contexto; se exploran los datos obtenidos, recurriendo a procesos estadísticos, y se deduce una serie de conclusiones con relación a la o las hipótesis}. (p. 97)
\end{displayquote}

\subsection{Método de investigación}

La investigación es de tipo aplicada

\subsection{Tipo de investigación}
La investigación es de tipo aplicada, que tiene por finalidad contribuir al conocimiento científicos de aprendizaje y herramientas pedagógicas. De acuerdo a textcite{santiago}.%\citeauthor{Duval} (1999, p. 15 citado en \citeauthor{ines}, \citeyear{ines}, p.~15).

\begin{displayquote}
\emph{Es también llamada práctica, empírica, activa y se encuentra íntimamente ligada a la investigación básica, ya que depende de sus descubrimientos y aportes teóricos para poder generar beneficios y bienestar a la sociedad. Se fundamenta en la investigación teórica; su finalidad especifica es aplicar las teorías existentes a la producción de normas y procedimientos tecnológicos para controlar situaciones  o procesos de la realidad.} (p.~15)
\end{displayquote}

Al respecto \cite{cerezal}, describe la investigación aplicada, como el manejo de las sapiencias, en la experiencia cotidiana, para emplearlos, en la totalidad de los casos, en beneficio de la humanidad.

SALDAÑA, J. (1998). ``Se refiere a un estudio de investigación en
el que se manipulan deliberadamente una o más variables independientes
(supuestas causas)
para analizar las consecuencias de esa manipulación sobre una o más
variables dependientes
(supuestos efectos), dentro de una situación de control para el investigador''.

%\subsubsection{Diseño de investigación}
%La investigación es de nivel explicativa experimental, porque se busca la causa y efecto en cada una de las variables de estudio, con manipulación de la variable independiente, es decir, se manipula la variable \MakeTextLowercase{\variablei}, para efectos o influencias en la \MakeTextLowercase{\variabled}.

%Por lo que \cite{khotari} explica que en una investigación experimental, de pruebas de hipótesis, cuando un grupo está expuesto a las condiciones habituales, se denomina \emph{grupo de control} (A), pero cuando lo esta en alguna condición nueva o especial, se denomina un \emph{grupo experimental } (B). Si los grupos A y B están expuestos a programas de estudios especiales, entonces ambos grupos se denominarían grupos experimentales. Es posible diseñar estudios que incluyan sólo grupos experimentales o estudios experimentales que incluyen tanto grupos experimentales como de control.

%In an experimental hypothesis-testing research when a
%group is exposed to usual conditions, it is termed a ‘control group’, but when the group is exposed to
%some novel or special condition, it is termed an ‘experimental group’. In the above illustration, the
%Group A can be called a control group and the Group B an experimental group. If both groups A and
%B are exposed to special studies programmes, then both groups would be termed ‘experimental
%groups.’ It is possible to design studies which include only experimental groups or studies which
%include both experimental and control groups.

\subsection{Diseño de investigación}
El diseño de investigación que se utilizó es el descriptivo – \MakeTextLowercase{\diseno}, que se estructura de acuerdo a la Figura \ref{figg}.
%individuos de la población iguales oportunidades de ser seleccionados, porque los grupos ya están formados con los estudiantes de la serie 200 en la \lugar.

\begin{figure}[ht!]\centering
	\begin{tikzpicture}[> = stealth,shorten > = 2pt,semithick]
	\node[] at (0,0)  (T){$M$};
	\node[] at (2,1.3)  (o1){$O_1$};
	\node[] at (5,1.3)  (w){$V_1$};
	\node[] at (2,0)  (r){$r$};
	\node[] at (2,-1.3)  (o2){$O_2$};
	\node[] at (5,-1.3)  (ww){$V_2$};
	\path[->] (T) edge node {} (o1);
	\path[->] (T) edge node {} (o2);
	\path[<-] (o1) edge node {} (w);
	\path[<-] (o2) edge node {} (ww);
	\path[->] (o1) edge node {} (r);
	\path[->] (o2) edge node {} (r);
	\end{tikzpicture}
	\caption{Diseño \MakeTextLowercase{\diseno}}
	\label{figg}
\end{figure}

Donde $M$ es la muestra, $O_1$ es la variable 1, \MakeTextLowercase{\variablei},
$O_2$ es la variable 2, \MakeTextLowercase{\variabled} y
$r$ es la relación entre variable 1 y variable 2.

El diseño de la investigación es cuasi -- experimental de un mismo grupo de trabajo con pre y pos -- prueba, porque se tomará un solo grupo experimental. De acuerdo a textcite{design} este diseño permitirá aplicar los módulos de experimentación en un periodo determinado y luego se trabajará de manera tradicional, alternando sucesivamente. Cuyo esquema de referencia es:

De acuerdo a :
\begin{displayquote}
\emph{www Al elegir un diseño se tiene en cuenta una secuencia de procesos para aplicarlos en un determinado problema y deducir resultados que verifiquen las hipótesis del problema. En este caso se tendrá el diseño propuesto con un solo grupo, que permitirá aplicar la primera variable sobre la segunda, para fortalecer la elaboración plástica tridimensional.}. \cite[p.~151]{inv1}
\end{displayquote}



\subsection{Población y muestra}

\subsubsection{Población}

Constituida por \poblacion, matriculados en el semestre impar del año 2019. %Los criterios que se considerarán en la selección de la muestra se describen en el Cuadro \ref{apt7}


\begin{table}[ht!]
\caption{Criterio de inclusión y exclusión}\label{apt7}
%\centering
\begin{tabular}{ccc}\Xhline{2pt}
Población&Aptos&No aptos\\\midrule
\makecell*[{{p{3.5cm}}}]{\centering Estudiantes matriculados del semestre impar} &
\makecell*[{{p{3.9cm}}}]{\centering Estudiantes regulares\\Asistentes puntuales} &\makecell[c]{ Estudiantes repitentes \\
   Estudiantes retirados \\
   Estudiantes del quinto superior\\
   Estudiantes no asistentes}\\\bottomrule
\end{tabular}
\end{table}


De acuerdo a \cite{inv1}, la población o universo es el agregado de todos los casos que concuerdan con determinadas descripciones o características, es decir es un conjunto de elementos que tiene al menos una misma característica que pueden ser medidos cuantitativamente o cualitativamente.

\subsubsection{Muestra}

En la presente investigación la muestra será no probabilística e intencional compuesta por un solo grupo experimental de \muestra, al cual se le aplicará el diseño \MakeTextLowercase{\diseno} de un mismo grupo con dos tipos de pruebas en series temporales equivalentes (Alternado).

Según \cite{bernal}, la muestra es una subcolección de elementos de la población seleccionada, de donde se obtendrá información para el desarrollo del estudio y del cual se hará medicines y observaciones de la variable dependiente e independiente, interrelacionada.

\subsubsection{Muestreo}

El muestreo será no probabilístico e intencional. Esta técnica consiste  en obtener una muestra donde los elementos  se recogen de manera aleatoria.

\cite{cordova} se refiere a la muestra no probabilística como aquella donde las unidades de la población tiene la misma posibilidad de ser elegidas y pertenecer a la muestra. Es deliberado, ya que esta práctica se opera en poblaciones indistintas. Aquí el especialista, conociendo la población y con buen juicio resuelve que características de análisis compondrá la muestra.


\subsection{Técnicas e instrumentos de recolección de datos}

\subsubsection{Técnica de recolección de datos}

\paragraph{Observación} Para el análisis de la influencia de la variable independiente sobre la variable dependiente se utilizó las técnicas de experimentación y observación, empleándose como instrumentos el plan de experimentación y la ficha de observación. Esta técnica que permitió recoger datos de la influencia de la variable independiente \MakeTextLowercase{\variablei} sobre la variable dependiente \MakeTextLowercase{\variabled}, en el proceso de experimentación.


%
Según \cite{khotari} el método de observación es el método más utilizado, especialmente en estudios relacionados con ciencias. La observación se convierte en una herramienta científica y en el método de recolección de datos para el investigador, cuando sirve a un propósito de investigación formulado, se planifica y registra sistemáticamente y se somete a controles de validez y fiabilidad.

\paragraph{Prueba pedagógica} Técnica que permitió recoger de manera escrita del  \MakeTextLowercase{\variablei} logro del aprendizaje de la variable dependiente \MakeTextLowercase{\variabled}, en el proceso de la experimentación

\cite{Livas} refiere: ``Es un proceso a través del cual se compara una unidad preestablecida y que la evaluación es un proceso que consiste en obtener infom1ación sistemática y objetiva acerca de un fenómeno e interpretar dicha información a fin de seleccionar entre distintas alternativas de decisión'' (p.~68).

\paragraph{Experimental} Técnica que permitió aplicar los módulos de experimentación de la variable \variablei.


{Según \cite{khotari} el método de observación es el método más utilizado, especialmente en estudios relacionados con ciencias. La observación se convierte en una herramienta científica y en el método de recolección de datos para el investigador, cuando sirve a un propósito de investigación formulado, se planifica y registra sistemáticamente y se somete a controles de validez y fiabilidad.}

\subsubsection{Instrumento de recolección de datos}

\paragraph{Plan de experimentación.} Documento pedagógico que contiene el proceso de aplicación de las estrategias pictóricas (ver anexo).

\paragraph{La ficha de observación} En el cual se incluyen indicadores que permitieron conocer  los indicadores de la variable independiente planteadas que influirán en el desarrollo de los indicadores de la variable dependiente en la muestra. En este caso el proceso será observado a través del evento práctico teórico que hicieron los observados.



%La ficha de observación por \cite{bernal}, es la exploración visual de lo que sucede en un contexto existente, en un fenómeno explícito, especificando y estableciendo los hechos acertados de acuerdo con algún diseño previsto, manipula un herramienta o prontuario impreso, reservado a conseguir respuestas sobre el problema en estudio.
%
%
%
\paragraph{La ficha de opinión}
%
Este instrumento permitirá recoger la opinion de los estudiantes con respecto al uso de la \MakeTextLowercase{\variablei}, y averiguar el impacto de esta. De acuerdo a \cite{opinion}. Una ficha de opinión, es una ficha en la que personas externas a la investigación escribe plasma lo que piensa en relación al texto o estudio que se está realizando.
%
%
%
Esta ficha puede tener: Número (es para tener orden de las fichas), Autor (la persona quien lo escribe), Tema de que se trata, Items o cuestionarios y Nota (en caso requerido).
%
%
%
%%URL del artículo: %http://www.ejemplode.com/13-ciencia/2411-ejemplo_de_ficha_de_opinion.html
%%Leer completo: ejemplos de Ficha de opinión
%%Los indicadores que conformaran la estructura de la ficha de
%%observación fueron elaborados a partir de la operacionalización
%%de la variable independiente, siendo las posibilidades de sus
%%respuestas: No, A veces y Sí.
\paragraph{Prueba escrita} Instrumento que permitirá recoger datos del desarrollo de las capacidades matemáticas  \MakeTextLowercase{\dimd},  \MakeTextLowercase{\dimdd}, \MakeTextLowercase{\dimddd} y \MakeTextLowercase{\dimddd}; instrumento elaborado de acuerdo a los indicadores descritos en el presente trabajo de investigación.

Las competencias y sus valoraciones cualitativas y cuantitativas se describen en el siguiente Cuadro \ref{evaluacion}.

\begin{table}[ht!]
%\centering
\caption{Criterio de calificación}\label{evaluacion}
\begin{tabular}{cccc}\Xhline{2pt}
\bf \multirow{2}*[-.01cm]{Capacidades}&\bf Valoración&\bf Valoración\\
&\bf  cualificada&\bf  cuantificada\\\midrule
\multirow{5}*[-.01cm]{\makecell*[{{p{7cm}}}]{\centering  \dimd  \dimdd \dimddd \dimddd}}&Excelente&17 -- 20\\
	&Bueno&13 -- 16\\	
	&Regular&09 -- 12\\	
	&Malo&05 -- 08\\
	&Deficiente&00 -- 04\\	
\bottomrule
\end{tabular}
\end{table}

%%  Es la forma de justipreciar las destrezas, conductas, contenidos, procurando un visto bueno, puntaje, calificativos o un concepto, este actúa como un dispositivo de revisión durante el proceso de enseñanza de ciertos indicadores establecidos y la exploración del efecto.
%%Se emplea, generalmente, para observar la conducta de los observados.
%%
%%
%%
%%Por \cite{inv1}, enuncia. Es un instrumento de medición cuyo propósito es que el estudiante demuestre la adquisición de un aprendizaje cognoscitivo, el dominio de una destreza o el desarrollo progresivo de una habilidad. Por su naturaleza, requiere respuesta escrita por parte del estudiante.
%
%\paragraph{Módulos de experimentación} Módulos que permitirán elaborar los materiales de aplicación de la variable \MakeTextLowercase{\variablei} con el objetivo de generar el \MakeTextLowercase{\variabled}. El plan experimentación se muestra en el Anexo 7.
%
%
%
%
%%Los indicadores que conforman la estructura de la Lista de cotejo
%%se obtuvieron a partir de la operacionalización de la variable
%%dependiente, siendo las posibilidades de sus respuestas: No, A veces y Sí.
%%Aplicado por los investigadores en las sesiones de aprendizajes y
%%con los resultados.
%
%

\subsection{Tratamiento estadístico de los datos}
Con el objetivo de procesar adecuadamente los datos y sacar conclusiones confiables, se utilizarán la las siguientes estadísticas.
 
\subsubsection{La estadística descriptiva} Esta estadística será de utilidad en la obtención de las siguientes, basadas en buscar el comportamiento de los datos, para inferir conclusiones relativas nuestras variables y la relación entre ellas.  
\begin{enumerate}
\item Tendencia central (media, mediana y moda),
\item dispersión o variabilidad (desviación estándar, la varianza y la regresión estándar),
\item curtosis (leptocurtica, mesocúrtica y platicúrtica) y 
\item asimetría (asimétrica positiva, simétrica y asimétrica negativa).
\end{enumerate}
\subsubsection{La estadística inferencial} Esta estadística se utilizara para probar, la normalidad y la homocedasticidad  de los datos, además de las prueba de hipótesis.

Cabe mencionar que se hará uso del software R Sweave y Excel para el procesamiento de los datos.

%\subsubsection{Validación de instrumentos}
%Consistirá en validar el contenido del instrumento por el criterio de jueces o expertos (con maestría o doctorado), quienes verificarán y evaluarán la coherencia y secuencialidad de los instrumentos.
%
%
%
%La opinión de los expertos consultados permitirá establecer la validez de los instrumentos que se empleará en la investigación. La validez se establecerá mediante el método del juicio de expertos.
%
%
%
%\begin{adjustwidth}{1cm}{}\gggl
%\emph{La validez, en términos generales, se refiere al grado en que un instrumento mide
%realmente la variable que pretende medir. Por ejemplo, un instrumento válido para medir la inteligencia debe medir la inteligencia y no la memoria. Un método para medir
%el rendimiento bursátil tiene que medir precisamente esto y no la imagen de una empresa. Un ejemplo aunque muy obvio de completa invalidez sería intentar medir el peso de los objetos con una cinta métrica en lugar de con una báscula.} \cite[p.~200]{inv1}
%\end{adjustwidth}

\subsection{Procedimientos}
\subsection{Estadísticos descriptivos}
\subsection{Estadísticos inferenciales}

Existen  diversos procedimientos para calcular la confiabilidad de un instrumento de medición. Todos utilizan formulas que producen coeficientes de confiabilidad. La mayoría de estos coeficientes pueden oscilar entre cero y uno donde un coeficiente cero significa nula confiabilidad y uno representa  un máximo de confiabilidad.

La confiabilidad de consistencia interna del instrumento, se determinará con la prueba piloto, en una muestra de 10 estudiantes que no serán miembros de la muestra, aplicando mitades de Coeficiente de Pearson y la corrección de Spearman Brow, la fórmula referencial es la siguiente: $$r_{xy}=\frac{n\Xsum xy-\cc{\Xsum x}\cc{\Xsum y}}{\sqrt{n\Xsum x^2-\cc{\Xsum x}^2}\sqrt{n\Xsum y^2-\cc{\Xsum y}^2}}$$

La corrección con Spearman Brow $$r=\frac{2r_{xy}}{1+r_{xy}}$$

	Donde:
$n$ tamaño de muestra
$x$ es la puntuación de la primera mitad (Impares)
$y$ es la puntuación de la segunda mitad (Pares)
$r_{xy}$ es el coeficiente Pearson
$r$ es el coeficiente de Spearman Brow. Si $0\leq r\leq 0.8$ se considera no confiable y si $0.8< r \leq1$ se considera confiable.

\begin{displayquote}
\emph{La confiabilidad de un instrumento de medición es el grado en que su aplicación
repetida al mismo individuo u objeto genera resultados similares. Por ejemplo, si se midiera en este momento la temperatura ambiental usando un termómetro y éste indicara que hay $22^\circ C$, y un minuto más tarde se consultara otra vez y señalara $5^\circ C$, tres minutos después se observara nuevamente y éste indicara $40^\circ C$, dicho termómetro no sería confiable, ya que su aplicación repetida produce resultados distintos. Asimismo, si una prueba de inteligencia (Intelligence Quotient, IQ) se aplica hoy a un grupo de personas y da ciertos valores de inteligencia, se aplica un mes después y proporciona valores diferentes, al igual que en subsecuentes mediciones, tal prueba no sería confiable. Los resultados no son coherentes, pues no se puede confiar en ellos.
} \cite[p.~61]{inv1}\end{displayquote}

\begin{table}[ht!]
% \centering\STautoround{3}
{%\tiny
\caption{Prueba de confiabilidad de la ficha de observación}\label{pret:1}
%\vspace{0.5cm}
\begin{tabular}{cccccccccccc}\Xhline{2pt}
AI&\multicolumn{10}{c}{INDICADORES}&$\sum$\\\midrule
&IN1&IN2&IN3&IN4&IN5&IN6&IN7&IN8&IN9&IN10&\\\cline{2-11}
1 &&&&&&&&&&&\\
2 &&&&&&&&&&&\\
3 &&&&&&&&&&&\\
4 &&&&&&&&&&&\\
5 &&&&&&&&&&&\\\midrule
$\sum$         &&&&&&&&&&&\\
$\overline{x}$ &&&&&&&&&&&\\
$s_i^2$        &&&&&&&&&&&\\
$s_i$          &&&&&&&&&&&\\\bottomrule
\end{tabular}\\\vspace{0.5cm}
{\normalsize Fuente: Datos obtenidos de la muestra}
}
\end{table}


El pre test con los datos obtenidos mediante la calificación de los expertos en artes plásticas la ficha de observación generó ... de acuerdo a la Tabla \ref{pret}, el instrumento (la ficha de observación)es confiable pues $0.8<\alpha<1$.

$$\sum s_i^2$$
$$s_t^2$$
$$\alpha=\frac{k}{k-1}\left(1-\frac{\sum s_i^2}{s_t^2}\right)$$
