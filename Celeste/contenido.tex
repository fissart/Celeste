%<*agradecimiento>
A la \lugar, a la conjunto de docentes, quienes en el transcurso de mis estudios.

durante los años de estudio supieron guiar mi formación profesional impartiendo sus conocimientos
artísticos.

Al docente \asesor, asesor de la esta investigación por su colaboración en la estructuración y elaboración 

A los docentes: \asesor, \expert y \expertt, quienes verificaron
y validaron los instrumentos de recolección de datos y las sugerencias
proporcionadas.

A los estudiantes y docentes de la \lugar, quienes posibilitaron la recolección de datos del presente trabajo de investigación.

A las personas y amistades que de algún modo colaboraron en la realización
de esta investigación.
%</agradecimiento>






%<*intro>
La investigación se realizó debido a que, en la \lugar, donde tradicionalmente
los docentes tienen poca creatividad, en cuanto al uso de los principios
de la composición plástica tridimensional en base a la geometría fractal,
implicando en el bajo rendimiento \emph{compositivo de los estudiantes.}
Se realiza esta investigación debido a la necesidad primordial de
mejorar, la metodología de los docentes, en la enseñanza de la composición
y colaborar con un modelo estructural didáctico más, entre muchos.
La característica abstracta de las geometría fractal no la excusa,
de hacerla puramente simbólica, es más esta es multidisciplinaria,
en la que participan diversos factores, entre las que se destaca el
aspecto visual de los objetos tridimensionales, con esta analogía
es posible descubrir más de lo evidente en las artes. Por tanto que
este aspecto colabore en la adquisición cognitiva de la composición
plástica tridimensional. En la enseñanza tradicional se obvia en demasía
la geometría fractal debido a las dificultades o falencias de los
docentes, probablemente por el bajo interes, de esta característica
en la enseñanza.

En las distintas carreras de la universidad se estudian teorías considerando
la representación del objeto estudiado, con el objetivo de descubrir
aspectos que las palabras o símbolos, no las pueden describir, lo
que implica considerar importante la representación y la estética
de los objetos tridimensionales, en la enseñanza aprendizaje de los
estudiantes universitarios, haciendo más ameno este proceso en su
vida educativa, favoreciendo el alcance de una actitud crítica de
lo estudiantes.

En resumen esta investigación se realizó porque se tuvo la necesidad
de proponer estrategias, en la solución de problemas de la composición
plástica tridimensional en el campo de las artes plásticas y generar
razones acerca de la utilidad y aplicabilidad de los resultados del
estudio. En particular \MakeTextLowercase{\objetivo}.

Resulta oportuno modificar nuestros métodos tradicionales de enseñanza
artística universitaria con fines de Acreditación y Certificación
de la Calidad Universitaria en la que estamos involucrados docentes,
estudiantes y comunidad universitaria los que nos exigen productos
de calidad, competentes, altamente eficientes y creativos.

Las razones expuestas, motivaron la realización del presente trabajo
investigación, titulada \MakeTextLowercase{\titulo}; donde las variables
de estudio son, la \MakeTextLowercase{\variablei} y el \MakeTextLowercase{\variabled},
los cuales se analizaron con el objetivo de lograr evaluar, la repercusión
de la primera en la segunda; se formuló el objetivo, \MakeTextLowercase{\objetivo},
a fin de contribuir en el campo del conocimiento artístico plástico
y la práctica constructiva, para mejorar los principios de la composición
plástica tridimensional: \MakeTextLowercase{\dimd}, \MakeTextLowercase{\dimdd},
\MakeTextLowercase{\dimddd}.
El fundamento teórico del presente trabajo de investigación, está
enmarcado con el enfoque, composición plástica para mejorar las competencias
plásticas compositivas. El contenido del presente trabajo de investigación
está estructurado en cuatro capítulos, en el primer capítulo trata
acerca del planteamiento del problema, el segundo se refiere al marco
teórico, el tercer capítulo concierne a la metodología de investigación,
el cuarto capítulo referido a los resultados de la investigación con
la discusión respectiva, finalmente conclusiones y sugerencias
%</intro>

%<*resumen>

El presente trabajo de investigación tuvo como objetivo \MakeTextLowercase{\objetivo}.
Tipo de investigación básica, nivel de investigación experimental
explicativa, de diseño cuasiexperimental de un grupo con pre y posprueba
en series temporales equivalentes (alternado); se empleó el método
hipotético deductivo, experimental y estadístico descriptivo e inferencial;
el lugar de estudio fue en la \lugar; la muestra fue no probabilística
e intencional compuesta por un solo grupo experimental de 35 estudiantes
de la serie 300, matriculados en el curso de arquitectura, del semestre
impar de la escuela profesional de Ingeniería Civil UNSCH; los datos
fueron recolectados a través de la prueba escrita y la ficha de observación;
la prueba de validez de instrumentos se realizó a través de juicio
de expertos y la confiabilidad, a través de prueba del Coeficiente
de Pearson y la corrección de Spearman Brow. Se verificó la no normalidad
de los datos, mediante la prueba de \emph{Shapiro Wilks}; se
aplicó la prueba de Student dos muestras relacionadas para la prueba
de hipótesis, con un nivel de confianza del 95\% y significancia del
5\% y se concluyó, que \MakeTextLowercase{\hipotesis}.

Palabras Claves: \variablei, \MakeTextLowercase{\variabled}, \dimi, \dimii, \dimd, \dimdd, \dimddd.
%</resumen>

%<*abstrac>
The objective of this investigation was to determine the influences
of the application of the fractal geometry in the plastic composition
of the students of The School of Professional Training of Civil Engineering
UNSCH 2019. Type of research. In the basics, level of explanatory
experimental research, of non-quasi-experimental design a group with
previous and subsequent tests in equivalent time series (alternate);
the job was hypothetical deductive, descriptive and descriptive and
inferential statistical method; the place of the study was carried
out in the School of Professional Training of Civil Engineering UNSCH
- Ayacucho; The sample was non-probabilistic and intentional composed
of a single experimental group of 35 students from the 300 series,
enrolled in the architecture course, from the odd semester. from the
UNSCH civil engineering professional school; The data was collected
through of the written test and the observation form; The validity
test of the instruments is performed. through expert judgment and
reliability, through the Pearson coefficient test and the correction
of Spearman's forehead. The non-normality of the data is verified,
through the Test of Shapiro Wilks; Two related samples are applied
to the student's test. for the hypothesis test, with a confidence
level of 95\% and a significance of 5\% and He concluded that the
application in fractal geometry significantly influences development.
Of the plastic composition of the students of the Training School.
Professional in Civil Engineering UNSCH 2019. 

Keywords: Fractal geometry, three-dimensional plastic composition, fractals
natural, abstract fractals, variety, unity, rhythm, balance and emphasis.
%</abstrac>








%<*planteamiento>
A nivel internacional, Uno de los problemas más comunes en la comunicación a nivel internacional es la barrera del idioma. Cuando las personas hablan diferentes idiomas, puede ser difícil entenderse mutuamente y transmitir información de manera efectiva. Esto puede conducir a malentendidos, confusiones e incluso conflictos.
Otro problema es la diferencia cultural. Cada país y cada región tiene sus propias normas, valores, creencias y formas de comunicación. Lo que puede ser considerado adecuado o normal en una cultura, puede ser ofensivo o inapropiado en otra. Esto puede generar malentendidos y hacer que las personas se sientan incomprendidas.

Además, la comunicación a nivel internacional también puede verse afectada por las diferencias en la forma de comunicarse. Algunas culturas son más directas y expresivas, mientras que otras son más indirectas y reservadas. Esto puede dificultar la interpretación de las intenciones y emociones detrás de las palabras y gestos.

También existe el problema de los estereotipos y prejuicios culturales. A menudo, las personas tienen ideas preconcebidas sobre otras culturas y países, lo que puede afectar la forma en que se comunican y se relacionan con los demás. Esto puede crear barreras en la comunicación y dificultar la construcción de relaciones sólidas y colaborativas.

En resumen, la comunicación acertiva a nivel internacional puede verse afectada por barreras del idioma, diferencias culturales, estilos de comunicación diferentes y estereotipos/prejuicios culturales. Para superar estos problemas, es importante tener una actitud abierta, flexible y respetuosa, y estar dispuesto a aprender sobre otras culturas y adaptarse a diferentes formas de comunicación.

A nivel nacional, la comunicación asertiva es un problema. Muchas personas tienen dificultades para expresar sus pensamientos y sentimientos de manera clara y respetuosa, lo que puede llevar a malentendidos y conflictos en las relaciones personales y profesionales.

Existen varios factores que contribuyen a este problema. En primer lugar, la educación en comunicación asertiva no siempre es prioritaria en el sistema escolar peruano. La mayoría de las personas aprenden a comunicarse a través de la observación y la experiencia, lo que puede resultar en patrones de comunicación poco saludables.

Además, en la cultura peruana existe una tendencia hacia la evitación del conflicto. Muchas personas prefieren evitar confrontaciones y problemas, lo que puede llevar a la falta de comunicación directa y honesta. En lugar de expresar sus necesidades y opiniones de manera clara, las personas suelen optar por la indirección y la pasividad.

Otro factor que contribuye al problema es la falta de habilidades de comunicación. Muchas personas no han sido enseñadas a escuchar activamente, a expresar sus pensamientos de manera clara y organizada, o a manejar conflictos de manera constructiva. Esto puede llevar a malentendidos, frustraciones y resentimientos en las relaciones interpersonales.

Para abordar este problema, es importante que se promueva la educación en comunicación asertiva desde la infancia. Las escuelas deben incluir programas que enseñen habilidades de comunicación, resolución de conflictos y empatía. Además, es necesario fomentar una cultura que valore la comunicación abierta, directa y respetuosa.

En resumen, la comunicación asertiva es un problema en el Perú debido a la falta de educación y habilidades en este ámbito, así como a la tendencia hacia la evitación del conflicto. Es importante promover la educación y la práctica de la comunicación asertiva para mejorar las relaciones personales y profesionales en el país.

A nivel regional, el problema de la comunicación asertiva en Ayacucho podría ser causado por varios factores, como la falta de educación en habilidades de comunicación, la barrera del idioma en comunidades indígenas, la falta de acceso a servicios de comunicación adecuados, entre otros.

En muchas áreas rurales de Ayacucho, es posible que las personas no hayan recibido una educación formal que incluya la enseñanza de habilidades de comunicación efectiva. Esto puede resultar en dificultades para expresarse claramente, entender a los demás y resolver conflictos de manera constructiva.

Además, Ayacucho es una región con una gran diversidad cultural, y muchas personas en comunidades indígenas pueden tener dificultades para comunicarse en el idioma dominante, como el español. Esto puede limitar su capacidad para expresarse y para entender y ser entendidos por los demás.
Otro problema es la falta de acceso a servicios de comunicación adecuados, como internet y telefonía celular. Esto puede dificultar la comunicación, especialmente en áreas rurales donde la infraestructura de comunicación es escasa.

Para abordar este problema, es importante proporcionar programas de educación en habilidades de comunicación en las escuelas y comunidades de Ayacucho. Además, es necesario fomentar la conservación de las lenguas indígenas y facilitar la traducción y el acceso a servicios de comunicación para todas las personas, incluso en áreas rurales.

A nivel local, El problema de la comunicación asertiva en los colegios de Ayacucho puede ser causado por varios factores. Algunos de estos factores pueden incluir:

\begin{enumerate}
	\item  Falta de habilidades de comunicación: Los estudiantes pueden no haber sido enseñados o no haber practicado habilidades de comunicación efectivas, lo que puede llevar a una comunicación no asertiva.
	\item  Barreras culturales: Las diferencias culturales pueden afectar la forma en que las personas se comunican y entienden los mensajes. Estas barreras pueden dificultar la comunicación asertiva entre diferentes grupos de estudiantes.
	\item  Falta de empatía: La falta de empatía puede dificultar la comprensión y empatía hacia los demás, lo que puede llevar a una comunicación no asertiva.
	\item  Ambiente inseguro: Un ambiente escolar inseguro puede hacer que los estudiantes se sientan inseguros o temerosos de expresar sus ideas o necesidades de manera asertiva.
	\item  Falta de enseñanza en habilidades de comunicación: Los colegios pueden no dar suficiente importancia a la enseñanza de habilidades de comunicación asertiva, lo que puede resultar en una falta de conocimiento sobre cómo comunicarse de manera efectiva.
	
\end{enumerate}

Para abordar este problema, es importante implementar programas y actividades que fomenten la comunicación asertiva en los colegios. Esto puede incluir talleres y sesiones de capacitación en habilidades de comunicación, promover un ambiente seguro y de apoyo para la expresión de ideas y sentimientos, y fomentar la empatía y la comprensión entre los estudiantes. También es esencial involucrar a los profesores y padres en el proceso, asegurándose de que también estén familiarizados con las habilidades de comunicación asertiva y puedan apoyar a los estudiantes en su desarrollo.

Por lo tanto el objetivo de la tesis es: Determinar como la caricatura influye para fortalecer la comunicación creativa en estudiantes del nivel secundario Ayacucho 2023. Pues Las caricaturas pueden ser una forma de comunicación muy efectiva, ya que utilizan imágenes y humor para transmitir mensajes y opiniones de manera más accesible y amigable. A través de las caricaturas, se pueden abordar temas complejos o controversiales de forma sencilla y divertida, lo que facilita que el mensaje llegue a un amplio público.

Las caricaturas también pueden ser utilizadas para la crítica social o política, utilizando personajes exagerados o estereotipados para resaltar problemas o situaciones específicas. Además, las caricaturas pueden ayudar a consolidar la identidad de una marca o de un producto, ya que a través de personajes y elementos visuales reconocibles, se puede transmitir la esencia de la empresa o del mensaje que se desea comunicar.

En resumen, las caricaturas son una forma de comunicación efectiva y atractiva, capaz de transmitir mensajes complejos o controversiales de manera accesible y divertida.
\cite{radford2017ensenanza}:


\begin{displayquote}
\emph{Las prácticas semióticas tienen como base los elementos críticos de una
actividad semiótica: elección del registro en el cual representar un objeto
matemático, capacidad de operar con las transformaciones de tratamiento
y de conversión. Dar un sentido coherente a cada representación semiótica
de un objeto matemático dado} (p.~57)
\end{displayquote}

En la \MakeTextLowercase{\lugar} los docentes
%</planteamiento>


%<*importancia>
El presente proyecto queda justificado  en el \lugar, por los siguientes ítems:
%</importancia>

%<*justificacionconcepto>
El presente proyecto queda justificado  en el \lugar, por los siguientes ítems:
%</justificacionconcepto>

%<*justificacionteo>
El principal objetivo de la  investigación es aportar al conocimiento existente en las didácticas matemáticas, esto es, sobre la relación de la \MakeTextLowercase{\variablei}, con las \MakeTextLowercase{\variabled}, cuyos resultados se podrá sistematizarse en una propuesta para ser incorporado como conocimiento a las ciencias de la educación matemática, ya que se estaría demostrando que existe relación directa y significativa entre estas dos variables de estudio.

%</justificacionteo>


%<*justificacionlegal>
La Constitución Política del Perú establece que la educación promueve el conocimiento y la práctica de las humanidades, la ciencia, la técnica, las artes, la educación física y el deporte, preparando para la vida y el trabajo y fomentando la solidaridad. El estado tiene la obligación de contribuir al desarrollo científico y tecnológico del país. La educación ética y cívica, así como la enseñanza de la Constitución y los derechos humanos, son obligatorias en todos los niveles educativos. La educación religiosa se imparte respetando la libertad de conciencia. Los medios de comunicación social deben cooperar con el estado en materia de educación, educación moral y cultural.

El derecho a la educación está reconocido por la Constitución Política y otras normativas nacionales, lo que implica que el Estado debe garantizar igualdad de oportunidades de acceso y retención en el sistema educativo, sin discriminación, y asegurar una educación de calidad. La Ley General de Educación Nº 28044 establece que la calidad de la educación es el nivel óptimo de formación que las personas deben alcanzar para enfrentar los desafíos del desarrollo humano, demostrando y continuando aprendiendo a lo largo de la vida. Además, se menciona la importancia de la investigación e innovación educativa, siendo responsabilidad del Ministerio de Educación promoverla dentro de su ámbito de competencia.
%</justificacionlegal>


%<*justificacioncientifica>
La caricatura es una forma de comunicación visual que utiliza elementos visuales como dibujos y textos para transmitir mensajes de manera humorística o satírica. Aunque puede parecer simplemente entretenimiento, la caricatura también tiene una base científica que justifica su uso como forma de comunicación.

En primer lugar, la caricatura utiliza elementos visuales que estimulan de manera efectiva el procesamiento cognitivo del cerebro. Nuestro cerebro procesa la información visual de manera más rápida y eficiente que la información verbal. Al combinar imágenes y texto, la caricatura aprovecha esta capacidad del cerebro para procesar la información de manera más efectiva, lo que facilita la transmisión del mensaje.

Además, la caricatura utiliza elementos de humor para transmitir mensajes de manera más efectiva. La risa y el humor tienen un impacto positivo en nuestro estado de ánimo y pueden mejorar la retención y comprensión de la información. Al utilizar el humor, la caricatura logra captar la atención del espectador y mantener su interés en el mensaje que se quiere transmitir.

Adicionalmente, la caricatura es una forma de comunicación que permite expresar ideas de manera indirecta o satírica. A través del uso de la exageración, la ironía o la parodia, la caricatura puede transmitir críticas sociales o políticas de manera sutil y efectiva. Esta capacidad de la caricatura de transmitir mensajes de manera indirecta es especialmente útil en situaciones en las que la comunicación directa puede ser censurada o restringida.

En resumen, la caricatura es una forma de comunicación que utiliza elementos visuales, humor y sátira para transmitir mensajes de manera efectiva. Aprovecha la capacidad cognitiva del cerebro para procesar la información visual, utiliza el humor para captar atención y mejorar la retención de la información, y permite expresar ideas de manera indirecta o satírica. Estas bases científicas justifican el uso de la caricatura como una forma válida de comunicación.
%</justificacioncientifica>

%<*justificacionsocial>
La caricatura es una forma de comunicación que se utiliza para transmitir mensajes de manera efectiva y humorística. Además de entretener, las caricaturas tienen una justificación social importante ya que pueden ser utilizadas para:

\begin{enumerate}
	\item  Crítica social: Las caricaturas suelen utilizar la exageración y la sátira para resaltar problemas y defectos de la sociedad. Pueden hacer críticas políticas, económicas, culturales y sociales, y de esta manera generar reflexión y conciencia en la audiencia.
	\item Denuncia de injusticias: A través de la caricatura se pueden denunciar injusticias, abusos y problemáticas sociales. Al exponer estas situaciones de manera cómica, se captura la atención del público y se fomenta el debate y la búsqueda de soluciones.
\item  Sensibilización: La caricatura puede ser utilizada para sensibilizar sobre problemas sociales o para llamar la atención sobre ciertos temas que necesitan ser abordados. Al usar imágenes impactantes o humorísticas, se logra captar la atención del público y generar empatía hacia ciertos asuntos.
\item  Promoción de valores: La caricatura también puede ser una herramienta para promover valores positivos y comportamientos éticos en la sociedad. Puede transmitir mensajes de respeto, tolerancia, igualdad, solidaridad, entre otros, y contribuir así a la construcción de una sociedad más justa y equitativa.
\end{enumerate}

En resumen, la justificación social de la caricatura en la comunicación radica en su capacidad para criticar, denunciar, sensibilizar y promover valores positivos en la sociedad. A través del humor y la sátira, pueden generar conciencia y promover el cambio social.
%</justificacionsocial>



%<*justificacionmet>
Una justificación metodológica de la caricatura como herramienta de comunicación puede basarse en los siguientes puntos:


\begin{enumerate}
\item  Simplificación visual: La caricatura simplifica los rasgos y características de una persona, objeto o situación, lo que permite transmitir un mensaje de manera rápida y clara. Al simplificar la imagen, la caricatura facilita la comprensión y la asimilación del mensaje por parte del espectador.
\item  Exageración: La exageración es una técnica muy común en la caricatura, ya que permite resaltar ciertos rasgos o aspectos de una persona o situación. La exageración ayuda a llamar la atención del espectador y a destacar los elementos más importantes o relevantes del mensaje.
\item  Humor y sátira: La caricatura busca transmitir mensajes de manera humorística o satírica, lo que ayuda a captar la atención del espectador y a generar una respuesta emocional. El humor y la sátira pueden hacer que el mensaje sea más memorable y pueden generar reflexión y debate en torno al tema tratado.
\item  Intencionalidad y contexto cultural: La caricatura se basa en la interpretación y el entendimiento del contexto cultural en el que se encuentra. Los elementos visuales, los colores, las situaciones y los personajes utilizados en la caricatura están diseñados de manera intencionada para transmitir un mensaje específico. La caricatura también está en constante diálogo con el entorno social y político en el que se encuentra, por lo que puede ser una respuesta a situaciones actuales o a temas de actualidad.
\item  Libertad creativa: La caricatura ofrece a los artistas una gran libertad creativa para transmitir sus ideas y mensajes. A través de la distorsión y la exageración, los caricaturistas pueden enfatizar ciertos aspectos y criticar o satirizar situaciones o personajes sin restricciones. Esta libertad permite que la caricatura sea una forma de comunicación visual muy poderosa y efectiva.
\end{enumerate}

En conclusión, la justificación metodológica de la caricatura como herramienta de comunicación radica en su capacidad para simplificar visualmente, exagerar rasgos, utilizar el humor y la sátira, establecer diálogo con el contexto cultural y ofrecer libertad creativa a los artistas. Estos elementos hacen de la caricatura una forma de comunicación visual muy efectiva para transmitir mensajes, ideas y críticas de manera rápida, clara y memorable.
%</justificacionmet>


%<*justificacionpra>
Aunque pueda parecer simple o superficial, la práctica de la caricatura tiene una serie de justificaciones prácticas que la hacen importante en el ámbito de la comunicación:

\begin{enumerate}
	\item  Síntesis visual: La caricatura condensa información en imágenes, permitiendo transmitir mensajes complejos de manera rápida y efectiva. A través de la exageración de rasgos característicos de una persona o situación, se simplifican conceptos y se facilita su comprensión.
\item  Crítica social: La caricatura es una herramienta para hacer una crítica constructiva de la realidad social y política. Mediante la sátira, se exponen problemas y se denuncian situaciones injustas o absurdas de forma humorística, lo que puede generar conciencia o reflexión en los espectadores.
\item  Libertad de expresión: La caricatura permite expresar opiniones de manera visual, sin necesidad de recurrir al lenguaje escrito. Esto es especialmente importante en contextos en los que la libertad de expresión puede estar limitada. Las imágenes pueden ser más poderosas y menos restrictivas que las palabras en algunos casos.
\item  Impacto emocional: Las caricaturas tienen la capacidad de generar emociones en los espectadores, ya sea a través de la risa, la indignación o la tristeza. Esta conexión emocional puede ser de gran importancia para movilizar a la gente y promover cambios sociales.
\item  Difusión masiva: La caricatura puede ser ampliamente difundida gracias a la facilidad de compartirla en redes sociales o imprimir en periódicos y revistas. Esto la convierte en un medio de comunicación de alcance masivo, capaz de llegar a un gran número de personas en poco tiempo.
\end{enumerate}

En resumen, la caricatura es una forma de comunicación práctica y efectiva, que sintetiza información, critica la realidad, se vale de la libertad de expresión, genera emociones y puede ser difundida masivamente. Todos estos elementos la convierten en una herramienta de gran valor en la comunicación contemporánea.
%</justificacionpra>

%<*justificacionped>
La aplicación de la \MakeTextLowercase{\variablei} en la didáctica en el desarrollo de las capacidades o competencias matemáticas se indaga mediante métodos científicos, situaciones que pueden ser investigadas por la ciencia, una vez que sean demostrados su validez y confiabilidad se podrán utilizar en otros trabajos de investigación y en diversas instituciones educativas.
%</justificacionped>


%<*justificacionsoc>
La investigación permitirá dar mayor importancia al desarrollo de la 
\MakeTextLowercase{\variablei} en los estudiantes de la \lugar y otras
instituciones similares en la región y el país. Asimismo será de utilidad para el
ejercicio de nuestro trabajo como docentes en secundaria y superior,
para orientar y brindar servicios de tutoría a las necesidades que presenten los estudiantes en el plano académico, social, y productivo mediante la prueba de la hipotesis: \MakeTextLowercase{\hipotesis}.
%</justificacionsoc>



%\subsection{Delimitación espacial y temporal}

%\subsubsection{Delimitación teórica}
%Este proyecto de investigación, se ejecutará en los estudiantes de la \lugar, cuyos resultados servirán de referencia en la práctica docente y externamente servirá de referencia para seguir colectivizando el uso conveniente del \MakeTextLowercase{\variablei} nivel regional y nacional.

%\subsubsection{Delimitación espacial}
%Este proyecto de investigación, se ejecutará en los estudiantes de la \lugar, cuyos resultados servirán de referencia en la práctica docente y externamente servirá de referencia para seguir colectivizando el uso conveniente del \MakeTextLowercase{\variablei} nivel regional y nacional.

%\subsubsection{Delimitación temporal}
%El presente proyecto de investigación se desenvolverá y se utilizará en los meses de abril a diciembre del año 2019, con los educandos de la serie 100 de la \lugar. En el distrito de Ayacucho 2019.

%\subsubsection{Delimitación metodológica}
%Al ser una investigación descriptiva correlacional y aplicada sólo a instituciones educativas de algunas localidades por provincia tiene dificultades para ser generalizados a otros departamentos y con algunos riesgos al departamento de Huancavelica. Asimismo el resultado de los datos estará en base de la sinceridad y estado de ánimo de los encuestados.



%<*internacional>

\cite{zurita_camacho_caricatura_2020} en la tesis  ``\usebibentry{zurita_camacho_caricatura_2020}{title}'', analiza el uso estratégico y comunicativo del personaje animado de Don Evaristo durante las administraciones municipales de Rodrigo Paz y Augusto Barrera, utilizando la metodología de la comunicación estratégica propuesta por Sandra Massoni.

Se utilizará la Versión Técnica Comunicacional (VTC) para analizar la frase núcleo del problema de comunicación, así como los componentes, síntomas y consecuencias relacionados con el espacio público urbano. También se realizará una identificación y jerarquización de los actores implicados en cada componente del problema comunicacional en ambas épocas.

Se analizarán las Marcas de Racionalidad Comunicacional propuestas por ambos personajes animados desde diferentes dimensiones de la comunicación. Además, se aplicarán indicadores de cambio social para evaluar el impacto e influencia de los procesos comunicativos, teniendo en cuenta la transformación social como respuesta a las características socioculturales locales.

El estudio se basará en entrevistas a personas clave involucradas en la creación de ambas caricaturas, con el fin de obtener distintos niveles de análisis, desde el estilo de gobierno de cada administración hasta las fortalezas y debilidades de cada caricatura en términos de comunicación estratégica. Se considerará a la comunicación estratégica como un fenómeno histórico, complejo, situacional y fluido que adopta diferentes configuraciones. 


\cite{segarra_morocho_caricatura_2021} en la tesis ``\usebibentry{segarra_morocho_caricatura_2021}{title}'', se estudió el sistema narrativo de la caricatura en tres de los principales medios impresos en Ecuador. Se empleó una metodología cualitativa a través del análisis de contenido de las caricaturas publicadas durante los meses de octubre, noviembre y diciembre de 2019.

Los resultados obtenidos revelaron que la caricatura es utilizada como una forma de comunicación gráfica que emite mensajes críticos sobre las situaciones sociales, políticas, económicas, etc., de la sociedad en general. Se demostró que el tema predominante de las caricaturas en los medios analizados es la política, seguido de la sociedad, y en menor medida, el deporte y otros temas.

En conclusión, la caricatura ha sido una forma de representación exagerada de los sucesos de la vida o de un personaje específico. Se ha utilizado como recurso de comunicación gráfica en los medios impresos y forma parte de los géneros de opinión de la prensa. El presente estudio contribuyó a comprender el sistema narrativo de la caricatura y los temas que predominan en los medios analizados en Ecuador.


\cite{bravo_quezada_alisis_2018} en la tesis  ``\usebibentry{bravo_quezada_alisis_2018}{title}'', estudia la tendencia política de los diarios locales a través de la caricatura como estrategia de comunicación política. Para ello, se tomaron en cuenta las caricaturas políticas de los candidatos locales que estaban en contienda por cargos públicos en la ciudad de Cuenca.

La investigación se llevó a cabo desde agosto, mes en el que inició la campaña electoral, hasta octubre, cuando se dio el cierre de la campaña en 2004. Debido a la falta de un método de análisis de caricatura existente, se creó una guía de análisis junto con un texto de apoyo que pueda servir como aporte para la Escuela de Comunicación Social de la Universidad de Cuenca.

El estudio se divide en siete capítulos. El primero aborda el lenguaje en la comunicación humana, mientras que el segundo se centra en la caricatura como estrategia de comunicación política. El tercer capítulo examina la caricatura política a lo largo de la historia, mientras que en el cuarto se realiza el análisis de las caricaturas de los dos diarios analizados.

El quinto capítulo se enfoca en la comunicación política a través de la caricatura, y el sexto considera a la caricatura como una expresión artística de comunicación política. Finalmente, el séptimo capítulo comprende el trabajo de campo, que consiste en el análisis de la caricatura en el público lector.

En las conclusiones y recomendaciones se exponen los hallazgos del estudio, así como las sugerencias para futuras investigaciones en el tema. Con esto, se espera contribuir al análisis de la tendencia política de los diarios locales a través de la caricatura como herramienta de comunicación política. 


%</internacional>

%<*nacional>

\cite{felix_seras_caricatura_2013} en la tesis  ``\usebibentry{felix_seras_caricatura_2013}{title}'',  la primera parte se define la caricatura política peruana como una modalidad narrativa de opinión, explicando sus características y analizando cómo se utiliza el humor y la iconografía en este tipo de relato.

En el segundo capítulo se analizan las características de la caricatura política peruana, así como las nuevas plataformas de comunicación en las que se ha desarrollado. Se realiza un recorrido histórico de la caricatura, utilizando imágenes de publicaciones y sucesos políticos peruanos para contextualizar al lector sobre el estilo iconográfico y las diferentes cargas valorativas utilizadas desde el siglo XIX hasta la actualidad.

En la tercera parte se analiza la producción de los caricaturistas peruanos Carlos "Carlín" Tovar, Eduardo "Heduardo" Rodríguez y Andrés Edery durante las elecciones municipales a la Alcaldía de Lima. Se muestra cómo su trabajo puede considerarse una modalidad narrativa de opinión dentro del periodismo.
Finalmente, en la última parte se exponen las conclusiones generales de la investigación.


\cite{vergel_loo_imagenes_2008} en la tesis ``\usebibentry{vergel_loo_imagenes_2008}{title}'', compara el uso estratégico y comunicativo del personaje animado Don Evaristo durante las administraciones municipales de Rodrigo Paz y Augusto Barrera en Quito. A través de la metodología de la comunicación estratégica propuesta por Sandra Massoni, se analiza cómo esta caricatura se utiliza como una herramienta de comunicación estratégica para promover el cambio social en el entorno urbano. Se utiliza la Versión Técnica Comunicacional para descomponer el problema de comunicación, identificando los componentes, síntomas y consecuencias relacionadas con el espacio público urbano. Se analiza también el papel de los actores implicados en cada componente del problema comunicacional en ambas épocas. Se examinan las Marcas de Racionalidad Comunicacional propuestas por ambos personajes animados, considerando las diferentes dimensiones de la comunicación. Además, se aplican indicadores de cambio social para medir el impacto e influencia de los procesos de comunicación desde una perspectiva transformadora que tenga en cuenta las particularidades socioculturales de la localidad. El estudio se apoya en entrevistas a personas clave involucradas en la creación de ambas caricaturas, proporcionando diferentes niveles de análisis, desde el estilo de gobierno de cada administración hasta las fortalezas y debilidades de cada caricatura en términos de comunicación estratégica.  

\cite{munoz_caricatura_2005} en su tesis ``\usebibentry{munoz_caricatura_2005}{title}'', se centra en el dibujo caricaturesco que se reprodujo y difundió en los medios gráficos masivos. Este género se basa en la deformación de los rasgos del rostro, reducción del cuerpo y la inclusión de objetos para resaltar las características de los personajes. El dibujo privilegia la observación cuidadosa de la estampa y los detalles, ya que una sola imagen debe transmitir toda la información necesaria. Además de la calidad de las obras, el valor de estas caricaturas radica en la cantidad de información que proporcionan sobre efemérides y el contexto en el que se crearon. Este género nos introduce en la experiencia humana a través de formas artísticas, linguísticas, religiosas, etc., y en el ámbito de la cultura política. La deformación de los rasgos humanos en este tipo de arte es una cualidad que solo los dibujantes talentosos pueden lograr, ya que requiere habilidades de las artes plásticas y una capacidad estética para experimentar y juzgar el mundo cotidiano. 

%</nacional>

%<*regional>


\cite{berrocal_vivanco_representacion_2015} en la tesis ``\usebibentry{berrocal_vivanco_representacion_2015}{title}'', analiza los dibujos de Guamán Poma de Ayala en su memorial Nueva Crónica y Buen Gobierno para demostrar la presencia dominante de la violencia simbólica. Se utilizó un enfoque sociocultural y se aplicaron metodologías de Hermenéutica y Semiología. Se identificaron dos tipos de violencia colonial: explícita y simbólica. Se encontró que la cultura dominante española utiliza la violencia simbólica para someter y empequeñecer a los indígenas, mientras que se representa a los españoles como poderosos y dominantes. La conclusión principal es que en los dibujos se representan la pasión, muerte y resurrección de Cristo para los indígenas, mientras que los españoles se representan como triunfantes y dominantes. 

\cite{yupanqui_flujos_2019} en su tesis ``\usebibentry{yupanqui_flujos_2019}{title}'', En conclusión, el estudio realizado mostró que la caricatura política fue capaz de resistir y adaptarse a la tensión generada por el conflicto armado interno en Perú. La caricatura fue utilizada como una herramienta más para transmitir información, criticar y denunciar, pero también mostró momentos de indiferencia y falta de compromiso hacia las causas estructurales del conflicto. En los años finales de la década de los ochenta, la caricatura política se volvió más lacónica y menos proactiva.




\cite{taipe_herreras_construccion_2015} en su tesis titulado ``\usebibentry{taipe_herreras_construccion_2015}{title}'', habla sobre estereotipos que presentan a la mujer peruana como objetos sexuales y perpetúan imágenes sexistas y discriminatorias. Además, se encontró que los medios impresos desempeñan un papel fundamental en la difusión y normalización de estos estereotipos al publicar y dar visibilidad a las caricaturas.

El análisis hermenéutico permitió interpretar el significado simbólico de las caricaturas, evidenciando cómo refuerzan la cosificación de la mujer y cómo se utilizan estereotipos de género para establecer roles sociales.

Por otro lado, el análisis hemerográfico reveló que estas caricaturas son publicadas con frecuencia en los diarios nacionales, lo que demuestra su presencia y alcance en la sociedad peruana. Asimismo, se encontró que estas caricaturas se alejan de los valores morales aceptados por la sociedad, perpetuando la desigualdad de género y la objetificación de la mujer.

En conclusión, este estudio muestra que las caricaturas de corte erótico presentes en los diarios nacionales contribuyen a la construcción de estereotipos de la mujer peruana, promoviendo la cosificación y la discriminación de género. Es importante generar conciencia sobre esta problemática y promover la inclusión y el respeto hacia la mujer en los medios de comunicación. 


%</regional>

%<*ov>
\textbf{\variablei} 
Para analizar las características del proceso de aplicación de la \MakeTextLowercase{\variablei}, se establecen como dimensiones la \emph{\MakeTextLowercase{\dimi}} y \emph{\MakeTextLowercase{\dimii}}, que cuentan con un total de 10 indicadores, que permitirán determinar el objetivo general esperado por estos procesos, por lo que se aplicara módulos de clases, talleres, exposición y discusión para mejorar el \MakeTextLowercase{\variabled}. Refiérase al Cuadro \ref{www}.

\begin{table}[ht!]
	\caption{Definición operacional de la variable independiente}
	\label{www}\begin{tabular}{ccccccc} 
			\Xhline{1pt} 		\textbf{\glstext{VD}}& \textbf{\glstext{D}} & \textbf{Indicadores}&\textbf{Items}&\textbf{Escala}& \textbf{Valores} \\ 		\hline 		\multirow{14}{*}{\rotatebox[origin=c]{90}
				{\makecell[{{p{5cm}}}]{\centering\variablei}}} 	
		&\multirow{3}{*}{\makecell[{{p{3cm}}}]{\ce\dimi}}
		& \multirow{1}{*}{\gb}&\multirow{1}{*}{P1--P2} 		&\multirow{14}{*}{\rotatebox[origin=c]{90}{\makecell{Cuantitativa\\ continua}}} 		&\multirow{14}{*}{\rotatebox[origin=c]{90}{\makecell{Valoración cuantitaitva \\ 0 a 20}}} 		\\\cline{3-4} 	
		&&\multirow{1}*{\gbb}&
		\multirow{1}{*}{C3--C4}&\\\cline{3-4}
		&&\multirow{1}*{\gbbb}&
		\multirow{1}{*}{C8}&\\\cline{2-4}
		&\multirow{4}*{\makecell[{{p{3cm}}}]{\ce\dimii}}
		&\multirow{1}*{\gbbbb}&
		\multirow{1}{*}{C9--C10}&\\\cline{3-4}
		&&\multirow{1}*{\gbbbbb}&
		\multirow{1}{*}{C11}&\\\cline{3-4}
		&&\multirow{1}*{\gbbbbbb}&
		\multirow{1}{*}{C12--C13}&\\\cline{3-4}
		&&\multirow{1}*{\gbbbbbbb}&
		\multirow{1}{*}{C14--C15}&\\\cline{3-4}
		\cline{2-4}
		&\multirow{4}*{\makecell[{{p{3cm}}}]{\ce\dimiii}}
		&\multirow{1}*{\gbbbbbbbbb}&
		\multirow{1}{*}{C9--C10}&\\\cline{3-4}
		&&\multirow{1}*{\gbbbbbbbbb}&
		\multirow{1}{*}{C11}&\\\cline{3-4}
		&&\multirow{1}*{\gbbbbbbbbbb}&
		\multirow{1}{*}{C12--C13}&\\\cline{3-4}
		&&\multirow{1}*{\gbbbbbbbbbbb}&
		\multirow{1}{*}{C12--C13}&\\\cline{3-4}
		\cline{2-4}
		&\multirow{3}*{\makecell[{{p{3cm}}}]{\ce\dimiiii}}
		& \multirow{1}*{\gbbbbbbbbbbbb}&
		\multirow{1}{*}{C9--C10}&\\\cline{3-4}
		&&\multirow{1}*{\gbbbbbbbbbbbbb}&
		\multirow{1}{*}{C11}&\\\cline{3-4}
		&&\multirow{1}*{\gbbbbbbbbbbbbbb}&
		\multirow{1}{*}{C14--C15}&\\
		\hline 
	\end{tabular}
\end{table}

\vspace{0.5cm}

\textbf{\variabled}

El presente trabajo de investigación, se refiere a la habilidad
de incrementar el \MakeTextLowercase{\variabled} lo cual involucra procesos o indicadores como la \MakeTextLowercase{\dimd}, \MakeTextLowercase{\dimdd} y  \MakeTextLowercase{\dimddd}. Se recogerá los datos usando la lista de cotejo, la fichas de observación y pruebas escritas. Para determinar el desarrollo de el \MakeTextLowercase{\variabled}, que cuentan con un total de 16 indicadores.
El cumplimiento determinará el crecimiento de la  eficacia del conocimiento matemático, refiérase al Cuadro \ref{vd}.

\begin{table}[ht!]
	\renewcommand\tabcolsep{0.15cm}\renewcommand\arraystretch{0.9}
	\caption{Definición operacional de la variable dependiente}
	\label{wwwww}
	\begin{tabular}{cccccc} 		\Xhline{1pt} 		\textbf{\glstext{VD}}& \textbf{\glstext{D}} & \textbf{Indicadores}&\textbf{Items}&\textbf{Escala}& \textbf{Valores} \\ 		\hline 		\multirow{9}{*}{\rotatebox[origin=c]{90}{\makecell[{{p{3.9cm}}}]{\ce\variabled}}} 	
		&\multirow{3}{*}{\makecell[{{p{3.9cm}}}]{\ce\dimd}} 	
		& \multirow{1}{*}{\fb}&\multirow{1}{*}{P1--P2} 		&\multirow{9}{*}{\rotatebox[origin=c]{90}{\makecell{Cuantitativa continua}}} 		&\multirow{9}{*}{\rotatebox[origin=c]{90}{\makecell{Valoración \\cuantitaitva de 0 a 20}}} 		\\\cline{3-4} 	
		&&\multirow{1}{*}{\fbb}&\multirow{1}{*}{P3--P4}&& 		\\\cline{3-4} 
		&&\multirow{1}{*}{\fbbb}&\multirow{1}{*}{P5}&&\\\cline{2-4} 	
		&\multirow{3}{*}{\makecell[{{p{3.9cm}}}]{\ce\dimdd}} 		& \multirow{1}{*}{\fbbbb}&\multirow{1}{*}{P6--P7}&& 		\\\cline{3-4} 	
		&&\multirow{1}{*}{\fbbbbb}&\multirow{1}{*}{P8}&& 		\\\cline{3-4}
		&&\multirow{1}{*}{\fbbbbbb}&\multirow{1}{*}{P9--P10}&& 		\\\cline{2-4} 	
		&\multirow{3}{*}{\makecell[{{p{3.9cm}}}]{\ce\dimddd}} 	
		&\multirow{1}{*}{\fbbbbbbb}&\multirow{1}{*}{P11--P12}&&\\\cline{3-4} 
		&&\multirow{1}{*}{\fbbbbbbbb}&\multirow{1}{*}{P13}&&\\\cline{3-4} 
		&&\multirow{1}{*}{\fbbbbbbbbb}&\multirow{1}{*}{P16--P17}&&\\	\Xhline{1pt}	\end{tabular}
\end{table}
%</ov>




%<*justificacionsoc>
%</justificacionsoc>

%<*justificacionsoc>
%</justificacionsoc>



\subsection{Enfoque de la investigación}
El presente trabajo de investigación se fundamenta en el enfoque cuantitativo, porque, utiliza la recolección de datos numéricos para probar hipótesis con base en la medición numérica y el análisis estadístico descriptivo e inferencial, con el fin establecer pautas de comportamiento y probar teorías.

De acuerdo a \cite{bernal} La orientación cuantitativa se fundamenta en el cálculo de características de los fenómenos sociales, lo cual presupone derivar de un cuadro conceptual adecuado al problema examinado, una cadena de proposiciones que enuncien relaciones entre si las variables experimentadas de forma justificada. Este procedimiento tiende a extender y sistematizar resultados. Según \cite{inv1}.

\begin{displayquote}
{La orientación cuantitativa se constituye, de un conjunto de términos que es secuencial y demostrativo. Cada fase antecede a la subsiguiente y no es posible omitir o evitar sucesos. La disposición es inflexible. Parte de una idea que va delimitándose y, una vez definida, se proceden a generar objetivos y cuestiones de exploración, se revisa la bibliografía y se elabora un marco o una configuración teórica. De las cuestiones se forman hipótesis y establecen variables; se bosqueja un procedimiento para experimentar la cual se conoce como diseño, se calculan las variables en un determinado contexto; se exploran los datos obtenidos, recurriendo a procesos estadísticos, y se deduce una serie de conclusiones con relación a la o las hipótesis}. (p. 97)
\end{displayquote}

\subsection{Método de investigación}

La investigación es de tipo aplicada

\subsection{Tipo de investigación}
La investigación es de tipo aplicada, que tiene por finalidad contribuir al conocimiento científicos de aprendizaje y herramientas pedagógicas. De acuerdo a textcite{santiago}.%\citeauthor{Duval} (1999, p. 15 citado en \citeauthor{ines}, \citeyear{ines}, p.~15).

\begin{displayquote}
\emph{Es también llamada práctica, empírica, activa y se encuentra íntimamente ligada a la investigación básica, ya que depende de sus descubrimientos y aportes teóricos para poder generar beneficios y bienestar a la sociedad. Se fundamenta en la investigación teórica; su finalidad especifica es aplicar las teorías existentes a la producción de normas y procedimientos tecnológicos para controlar situaciones  o procesos de la realidad.} (p.~15)
\end{displayquote}

Al respecto \cite{cerezal}, describe la investigación aplicada, como el manejo de las sapiencias, en la experiencia cotidiana, para emplearlos, en la totalidad de los casos, en beneficio de la humanidad.

SALDAÑA, J. (1998). ``Se refiere a un estudio de investigación en
el que se manipulan deliberadamente una o más variables independientes
(supuestas causas)
para analizar las consecuencias de esa manipulación sobre una o más
variables dependientes
(supuestos efectos), dentro de una situación de control para el investigador''.

%\subsubsection{Diseño de investigación}
%La investigación es de nivel explicativa experimental, porque se busca la causa y efecto en cada una de las variables de estudio, con manipulación de la variable independiente, es decir, se manipula la variable \MakeTextLowercase{\variablei}, para efectos o influencias en la \MakeTextLowercase{\variabled}.

%Por lo que \cite{khotari} explica que en una investigación experimental, de pruebas de hipótesis, cuando un grupo está expuesto a las condiciones habituales, se denomina \emph{grupo de control} (A), pero cuando lo esta en alguna condición nueva o especial, se denomina un \emph{grupo experimental } (B). Si los grupos A y B están expuestos a programas de estudios especiales, entonces ambos grupos se denominarían grupos experimentales. Es posible diseñar estudios que incluyan sólo grupos experimentales o estudios experimentales que incluyen tanto grupos experimentales como de control.

%In an experimental hypothesis-testing research when a
%group is exposed to usual conditions, it is termed a ‘control group’, but when the group is exposed to
%some novel or special condition, it is termed an ‘experimental group’. In the above illustration, the
%Group A can be called a control group and the Group B an experimental group. If both groups A and
%B are exposed to special studies programmes, then both groups would be termed ‘experimental
%groups.’ It is possible to design studies which include only experimental groups or studies which
%include both experimental and control groups.

\subsection{Diseño de investigación}
El diseño de investigación que se utilizó es el descriptivo – \MakeTextLowercase{\diseno}, que se estructura de acuerdo a la Figura \ref{figg}.
%individuos de la población iguales oportunidades de ser seleccionados, porque los grupos ya están formados con los estudiantes de la serie 200 en la \lugar.

\begin{figure}[ht!]\centering
	\begin{tikzpicture}[> = stealth,shorten > = 2pt,semithick]
	\node[] at (0,0)  (T){$M$};
	\node[] at (2,1.3)  (o1){$O_1$};
	\node[] at (5,1.3)  (w){$V_1$};
	\node[] at (2,0)  (r){$r$};
	\node[] at (2,-1.3)  (o2){$O_2$};
	\node[] at (5,-1.3)  (ww){$V_2$};
	\path[->] (T) edge node {} (o1);
	\path[->] (T) edge node {} (o2);
	\path[<-] (o1) edge node {} (w);
	\path[<-] (o2) edge node {} (ww);
	\path[->] (o1) edge node {} (r);
	\path[->] (o2) edge node {} (r);
	\end{tikzpicture}
	\caption{Diseño \MakeTextLowercase{\diseno}}
	\label{figg}
\end{figure}

Donde $M$ es la muestra, $O_1$ es la variable 1, \MakeTextLowercase{\variablei},
$O_2$ es la variable 2, \MakeTextLowercase{\variabled} y
$r$ es la relación entre variable 1 y variable 2.

El diseño de la investigación es cuasi -- experimental de un mismo grupo de trabajo con pre y pos -- prueba, porque se tomará un solo grupo experimental. De acuerdo a textcite{design} este diseño permitirá aplicar los módulos de experimentación en un periodo determinado y luego se trabajará de manera tradicional, alternando sucesivamente. Cuyo esquema de referencia es:

De acuerdo a :
\begin{displayquote}
\emph{www Al elegir un diseño se tiene en cuenta una secuencia de procesos para aplicarlos en un determinado problema y deducir resultados que verifiquen las hipótesis del problema. En este caso se tendrá el diseño propuesto con un solo grupo, que permitirá aplicar la primera variable sobre la segunda, para fortalecer la elaboración plástica tridimensional.}. \cite[p.~151]{inv1}
\end{displayquote}



\subsection{Población y muestra}

\subsubsection{Población}

Constituida por \poblacion, matriculados en el semestre impar del año 2019. %Los criterios que se considerarán en la selección de la muestra se describen en el Cuadro \ref{apt7}


\begin{table}[ht!]
\caption{Criterio de inclusión y exclusión}\label{apt7}
%\centering
\begin{tabular}{ccc}\Xhline{2pt}
Población&Aptos&No aptos\\\midrule
\makecell*[{{p{3.5cm}}}]{\centering Estudiantes matriculados del semestre impar} &
\makecell*[{{p{3.9cm}}}]{\centering Estudiantes regulares\\Asistentes puntuales} &\makecell[c]{ Estudiantes repitentes \\
   Estudiantes retirados \\
   Estudiantes del quinto superior\\
   Estudiantes no asistentes}\\\bottomrule
\end{tabular}
\end{table}


De acuerdo a \cite{inv1}, la población o universo es el agregado de todos los casos que concuerdan con determinadas descripciones o características, es decir es un conjunto de elementos que tiene al menos una misma característica que pueden ser medidos cuantitativamente o cualitativamente.

\subsubsection{Muestra}

En la presente investigación la muestra será no probabilística e intencional compuesta por un solo grupo experimental de \muestra, al cual se le aplicará el diseño \MakeTextLowercase{\diseno} de un mismo grupo con dos tipos de pruebas en series temporales equivalentes (Alternado).

Según \cite{bernal}, la muestra es una subcolección de elementos de la población seleccionada, de donde se obtendrá información para el desarrollo del estudio y del cual se hará medicines y observaciones de la variable dependiente e independiente, interrelacionada.

\subsubsection{Muestreo}

El muestreo será no probabilístico e intencional. Esta técnica consiste  en obtener una muestra donde los elementos  se recogen de manera aleatoria.

\cite{cordova} se refiere a la muestra no probabilística como aquella donde las unidades de la población tiene la misma posibilidad de ser elegidas y pertenecer a la muestra. Es deliberado, ya que esta práctica se opera en poblaciones indistintas. Aquí el especialista, conociendo la población y con buen juicio resuelve que características de análisis compondrá la muestra.


\subsection{Técnicas e instrumentos de recolección de datos}

\subsubsection{Técnica de recolección de datos}

\paragraph{Observación} Para el análisis de la influencia de la variable independiente sobre la variable dependiente se utilizó las técnicas de experimentación y observación, empleándose como instrumentos el plan de experimentación y la ficha de observación. Esta técnica que permitió recoger datos de la influencia de la variable independiente \MakeTextLowercase{\variablei} sobre la variable dependiente \MakeTextLowercase{\variabled}, en el proceso de experimentación.


%
Según \cite{khotari} el método de observación es el método más utilizado, especialmente en estudios relacionados con ciencias. La observación se convierte en una herramienta científica y en el método de recolección de datos para el investigador, cuando sirve a un propósito de investigación formulado, se planifica y registra sistemáticamente y se somete a controles de validez y fiabilidad.

\paragraph{Prueba pedagógica} Técnica que permitió recoger de manera escrita del  \MakeTextLowercase{\variablei} logro del aprendizaje de la variable dependiente \MakeTextLowercase{\variabled}, en el proceso de la experimentación

\cite{Livas} refiere: ``Es un proceso a través del cual se compara una unidad preestablecida y que la evaluación es un proceso que consiste en obtener infom1ación sistemática y objetiva acerca de un fenómeno e interpretar dicha información a fin de seleccionar entre distintas alternativas de decisión'' (p.~68).

\paragraph{Experimental} Técnica que permitió aplicar los módulos de experimentación de la variable \variablei.


{Según \cite{khotari} el método de observación es el método más utilizado, especialmente en estudios relacionados con ciencias. La observación se convierte en una herramienta científica y en el método de recolección de datos para el investigador, cuando sirve a un propósito de investigación formulado, se planifica y registra sistemáticamente y se somete a controles de validez y fiabilidad.}

\subsubsection{Instrumento de recolección de datos}

\paragraph{Plan de experimentación.} Documento pedagógico que contiene el proceso de aplicación de las estrategias pictóricas (ver anexo).

\paragraph{La ficha de observación} En el cual se incluyen indicadores que permitieron conocer  los indicadores de la variable independiente planteadas que influirán en el desarrollo de los indicadores de la variable dependiente en la muestra. En este caso el proceso será observado a través del evento práctico teórico que hicieron los observados.



%La ficha de observación por \cite{bernal}, es la exploración visual de lo que sucede en un contexto existente, en un fenómeno explícito, especificando y estableciendo los hechos acertados de acuerdo con algún diseño previsto, manipula un herramienta o prontuario impreso, reservado a conseguir respuestas sobre el problema en estudio.
%
%
%
\paragraph{La ficha de opinión}
%
Este instrumento permitirá recoger la opinion de los estudiantes con respecto al uso de la \MakeTextLowercase{\variablei}, y averiguar el impacto de esta. De acuerdo a \cite{opinion}. Una ficha de opinión, es una ficha en la que personas externas a la investigación escribe plasma lo que piensa en relación al texto o estudio que se está realizando.
%
%
%
Esta ficha puede tener: Número (es para tener orden de las fichas), Autor (la persona quien lo escribe), Tema de que se trata, Items o cuestionarios y Nota (en caso requerido).
%
%
%
%%URL del artículo: %http://www.ejemplode.com/13-ciencia/2411-ejemplo_de_ficha_de_opinion.html
%%Leer completo: ejemplos de Ficha de opinión
%%Los indicadores que conformaran la estructura de la ficha de
%%observación fueron elaborados a partir de la operacionalización
%%de la variable independiente, siendo las posibilidades de sus
%%respuestas: No, A veces y Sí.
\paragraph{Prueba escrita} Instrumento que permitirá recoger datos del desarrollo de las capacidades matemáticas  \MakeTextLowercase{\dimd},  \MakeTextLowercase{\dimdd}, \MakeTextLowercase{\dimddd} y \MakeTextLowercase{\dimddd}; instrumento elaborado de acuerdo a los indicadores descritos en el presente trabajo de investigación.

Las competencias y sus valoraciones cualitativas y cuantitativas se describen en el siguiente Cuadro \ref{evaluacion}.

\begin{table}[ht!]
%\centering
\caption{Criterio de calificación}\label{evaluacion}
\begin{tabular}{cccc}\Xhline{2pt}
\bf \multirow{2}*[-.01cm]{Capacidades}&\bf Valoración&\bf Valoración\\
&\bf  cualificada&\bf  cuantificada\\\midrule
\multirow{5}*[-.01cm]{\makecell*[{{p{7cm}}}]{\centering  \dimd  \dimdd \dimddd \dimddd}}&Excelente&17 -- 20\\
	&Bueno&13 -- 16\\	
	&Regular&09 -- 12\\	
	&Malo&05 -- 08\\
	&Deficiente&00 -- 04\\	
\bottomrule
\end{tabular}
\end{table}

%%  Es la forma de justipreciar las destrezas, conductas, contenidos, procurando un visto bueno, puntaje, calificativos o un concepto, este actúa como un dispositivo de revisión durante el proceso de enseñanza de ciertos indicadores establecidos y la exploración del efecto.
%%Se emplea, generalmente, para observar la conducta de los observados.
%%
%%
%%
%%Por \cite{inv1}, enuncia. Es un instrumento de medición cuyo propósito es que el estudiante demuestre la adquisición de un aprendizaje cognoscitivo, el dominio de una destreza o el desarrollo progresivo de una habilidad. Por su naturaleza, requiere respuesta escrita por parte del estudiante.
%
%\paragraph{Módulos de experimentación} Módulos que permitirán elaborar los materiales de aplicación de la variable \MakeTextLowercase{\variablei} con el objetivo de generar el \MakeTextLowercase{\variabled}. El plan experimentación se muestra en el Anexo 7.
%
%
%
%
%%Los indicadores que conforman la estructura de la Lista de cotejo
%%se obtuvieron a partir de la operacionalización de la variable
%%dependiente, siendo las posibilidades de sus respuestas: No, A veces y Sí.
%%Aplicado por los investigadores en las sesiones de aprendizajes y
%%con los resultados.
%
%

\subsection{Tratamiento estadístico de los datos}
Con el objetivo de procesar adecuadamente los datos y sacar conclusiones confiables, se utilizarán la las siguientes estadísticas.
 
\subsubsection{La estadística descriptiva} Esta estadística será de utilidad en la obtención de las siguientes, basadas en buscar el comportamiento de los datos, para inferir conclusiones relativas nuestras variables y la relación entre ellas.  
\begin{enumerate}
\item Tendencia central (media, mediana y moda),
\item dispersión o variabilidad (desviación estándar, la varianza y la regresión estándar),
\item curtosis (leptocurtica, mesocúrtica y platicúrtica) y 
\item asimetría (asimétrica positiva, simétrica y asimétrica negativa).
\end{enumerate}
\subsubsection{La estadística inferencial} Esta estadística se utilizara para probar, la normalidad y la homocedasticidad  de los datos, además de las prueba de hipótesis.

Cabe mencionar que se hará uso del software R Sweave y Excel para el procesamiento de los datos.

%\subsubsection{Validación de instrumentos}
%Consistirá en validar el contenido del instrumento por el criterio de jueces o expertos (con maestría o doctorado), quienes verificarán y evaluarán la coherencia y secuencialidad de los instrumentos.
%
%
%
%La opinión de los expertos consultados permitirá establecer la validez de los instrumentos que se empleará en la investigación. La validez se establecerá mediante el método del juicio de expertos.
%
%
%
%\begin{adjustwidth}{1cm}{}\gggl
%\emph{La validez, en términos generales, se refiere al grado en que un instrumento mide
%realmente la variable que pretende medir. Por ejemplo, un instrumento válido para medir la inteligencia debe medir la inteligencia y no la memoria. Un método para medir
%el rendimiento bursátil tiene que medir precisamente esto y no la imagen de una empresa. Un ejemplo aunque muy obvio de completa invalidez sería intentar medir el peso de los objetos con una cinta métrica en lugar de con una báscula.} \cite[p.~200]{inv1}
%\end{adjustwidth}

\subsection{Procedimientos}
\subsection{Estadísticos descriptivos}
\subsection{Estadísticos inferenciales}

Existen  diversos procedimientos para calcular la confiabilidad de un instrumento de medición. Todos utilizan formulas que producen coeficientes de confiabilidad. La mayoría de estos coeficientes pueden oscilar entre cero y uno donde un coeficiente cero significa nula confiabilidad y uno representa  un máximo de confiabilidad.

La confiabilidad de consistencia interna del instrumento, se determinará con la prueba piloto, en una muestra de 10 estudiantes que no serán miembros de la muestra, aplicando mitades de Coeficiente de Pearson y la corrección de Spearman Brow, la fórmula referencial es la siguiente: $$r_{xy}=\frac{n\Xsum xy-\cc{\Xsum x}\cc{\Xsum y}}{\sqrt{n\Xsum x^2-\cc{\Xsum x}^2}\sqrt{n\Xsum y^2-\cc{\Xsum y}^2}}$$

La corrección con Spearman Brow $$r=\frac{2r_{xy}}{1+r_{xy}}$$

	Donde:
$n$ tamaño de muestra
$x$ es la puntuación de la primera mitad (Impares)
$y$ es la puntuación de la segunda mitad (Pares)
$r_{xy}$ es el coeficiente Pearson
$r$ es el coeficiente de Spearman Brow. Si $0\leq r\leq 0.8$ se considera no confiable y si $0.8< r \leq1$ se considera confiable.

\begin{displayquote}
\emph{La confiabilidad de un instrumento de medición es el grado en que su aplicación
repetida al mismo individuo u objeto genera resultados similares. Por ejemplo, si se midiera en este momento la temperatura ambiental usando un termómetro y éste indicara que hay $22^\circ C$, y un minuto más tarde se consultara otra vez y señalara $5^\circ C$, tres minutos después se observara nuevamente y éste indicara $40^\circ C$, dicho termómetro no sería confiable, ya que su aplicación repetida produce resultados distintos. Asimismo, si una prueba de inteligencia (Intelligence Quotient, IQ) se aplica hoy a un grupo de personas y da ciertos valores de inteligencia, se aplica un mes después y proporciona valores diferentes, al igual que en subsecuentes mediciones, tal prueba no sería confiable. Los resultados no son coherentes, pues no se puede confiar en ellos.
} \cite[p.~61]{inv1}\end{displayquote}

\begin{table}[ht!]
% \centering\STautoround{3}
{%\tiny
\caption{Prueba de confiabilidad de la ficha de observación}\label{pret:1}
%\vspace{0.5cm}
\begin{tabular}{cccccccccccc}\Xhline{2pt}
AI&\multicolumn{10}{c}{INDICADORES}&$\sum$\\\midrule
&IN1&IN2&IN3&IN4&IN5&IN6&IN7&IN8&IN9&IN10&\\\cline{2-11}
1 &&&&&&&&&&&\\
2 &&&&&&&&&&&\\
3 &&&&&&&&&&&\\
4 &&&&&&&&&&&\\
5 &&&&&&&&&&&\\\midrule
$\sum$         &&&&&&&&&&&\\
$\overline{x}$ &&&&&&&&&&&\\
$s_i^2$        &&&&&&&&&&&\\
$s_i$          &&&&&&&&&&&\\\bottomrule
\end{tabular}\\\vspace{0.5cm}
{\normalsize Fuente: Datos obtenidos de la muestra}
}
\end{table}


El pre test con los datos obtenidos mediante la calificación de los expertos en artes plásticas la ficha de observación generó ... de acuerdo a la Tabla \ref{pret}, el instrumento (la ficha de observación)es confiable pues $0.8<\alpha<1$.

$$\sum s_i^2$$
$$s_t^2$$
$$\alpha=\frac{k}{k-1}\left(1-\frac{\sum s_i^2}{s_t^2}\right)$$
