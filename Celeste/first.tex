\usepackage{textcase}
\newcommand{\autor}{Bach. Quispe Urbano, Luzmerly Celeste\xspace}
\newcommand{\asesor}{Lic. Canchari Soliz, Jose\xspace}

\newcommand{\expert}{\xspace}
\newcommand{\expertt}{\xspace}

\newcommand{\muestra}{20 estudiantes del colegio público \NoCaseChange{S}an \NoCaseChange{J}uán \NoCaseChange{A}yacucho\xspace}

\newcommand{\poblacion}{900 estudiantes del colegio público \NoCaseChange{S}an \NoCaseChange{J}uán \NoCaseChange{A}yacucho\xspace}

\newcommand{\lugar}{colegio publico \NoCaseChange{S}an \NoCaseChange{J}uan \NoCaseChange{A}yacucho\xspace}
%TITULO

\newcommand{\titulo}{LA CARICATURA PARA FORTALECER LA COMUNICACIÓN ASERTIVA EN ESTUDIANTES DE EDUCACIÓN SECUNDARIA DE HUAMANGA, AYACUCHO - 2023\xspace}
% VARIABLES
\newcommand{\variablei}{{{Caricatura}}\xspace}
\newcommand{\variabled}{{{Comunicación asertiva}}\xspace} 
% DIMENSIONES
% VI
\newcommand{\dimi}{{{Realidad}}\xspace}
\newcommand{\dimii}{{{Fantasía}}\xspace}
\newcommand{\dimiii}{{{Política}}\xspace}
\newcommand{\dimiiii}{{{Periodística}}\xspace}
% VD 
\newcommand{\dimd}{Empatía\xspace}
\newcommand{\dimdd}{Asertividad\xspace}
\newcommand{\dimddd}{Atención\xspace}
\newcommand{\dimdddd}{Pensamiento\xspace}
%%%%%%%%%%%%% Metodologia
\newcommand{\enfoque}{\emph{Cuantitativa}\xspace}
\newcommand{\tipo}{\emph{Pre-experimental}\xspace}
\newcommand{\nivel}{\emph{Explicativa}\xspace}
\newcommand{\diseno}{\emph{Experimental}\xspace}

%%%%%%%%%%% INDICADORES

% INDICADORES DE LA VARIABLE INDEPENDIENTE

% Inteligencia espacial
\newcommand{\gb}{Construye objetos asociadas\xspace}
\newcommand{\gbb}{Manipula objetos asociadas\xspace}
\newcommand{\gbbb}{Asocia objetos reales\xspace}

\newcommand{\gbbbb}{wwww\xspace}
\newcommand{\gbbbbb}{w5\xspace}
\newcommand{\gbbbbbb}{w6\xspace}
\newcommand{\gbbbbbbb}{w7\xspace}

\newcommand{\gbbbbbbbb}{w8\xspace}
\newcommand{\gbbbbbbbbb}{w9\xspace}
\newcommand{\gbbbbbbbbbb}{w10\xspace}
\newcommand{\gbbbbbbbbbbb}{w11\xspace}

\newcommand{\gbbbbbbbbbbbb}{w12\xspace}
\newcommand{\gbbbbbbbbbbbbb}{w13\xspace}
\newcommand{\gbbbbbbbbbbbbbb}{w14\xspace}

% INDICADORES DE LA VARIABLE DEPENDIENTE

\newcommand{\fb}{1\xspace}
\newcommand{\fbb}{2\xspace}
\newcommand{\fbbb}{3\xspace}
\newcommand{\fbbbb}{4\xspace}
\newcommand{\fbbbbb}{5\xspace}
\newcommand{\fbbbbbb}{6\xspace}
\newcommand{\fbbbbbbb}{7\xspace}
\newcommand{\fbbbbbbbb}{8\xspace}
\newcommand{\fbbbbbbbbb}{9\xspace}

% PROBLEMA
\newcommand{\problema}{¿En qué medida influye la caricatura para fortalecer la comunicación creativa en estudiantes del nivel secundario \NoCaseChange{A}yacucho 2023?\xspace}

\newcommand{\problemae}{¿En qué proporción influye la caricatura para fortalecer la empatía en la comunicación creativa?\xspace}

\newcommand{\problemaee}{¿De qué manera influye la caricatura para fortalecer la asertividad en la comunicación creativa?\xspace}

\newcommand{\problemaeee}{¿En qué proporción influye la caricatura para fortalecer la atención en la comunicación creativa?\xspace}

\newcommand{\problemaeeee}{¿En qué medida influye la caricatura para fortalecer el pensamiento en la comunicación creativa?.\xspace}



% OBJETIVO
\newcommand{\objetivo}{Determinar como la caricatura influye para fortalecer la comunicación creativa en estudiantes del nivel secundario \NoCaseChange{A}yacucho 2023.\xspace}

\newcommand{\objetivoe}{Determinar como la caricatura influye para fortalecer la empatía en la comunicación creativa\xspace}

\newcommand{\objetivoee}{Establecer como la caricatura influye para fortalecer la asertividad en la comunicación creativa\xspace}

\newcommand{\objetivoeee}{Precisar como la caricatura influye para fortalecer la atención en la comunicación creativa\xspace}

\newcommand{\objetivoeeee}{Determinar como la caricatura influye para fortalecer el pensamiento en la comunicación creativa\xspace}



% HIPOTESIS
\newcommand{\hipotesis}{Existe influencia significativa en la caricatura para fortalecer la comunicación creativa en estudiantes del nivel secundario Ayacucho 2023\xspace}

\newcommand{\hipotesise}{Se establece la influencia significativa en la caricatura para fortalecer la empatía en la comunicación creativa\xspace}

\newcommand{\hipotesisee}{Se corrobora la influencia significativa en la caricatura para fortalecer la asertividad en la comunicación creativa\xspace}

\newcommand{\hipotesiseee}{Se evidencia influencia significativa en la caricatura para fortalecer la atención en la comunicación creativa\xspace}

\newcommand{\hipotesiseeee}{Se establece influencia significativa en la caricatura para fortalecer el pensamiento en la comunicación creativa\xspace}




\newcommand{\N}{\mathds{N}}
\newcommand{\R}{\mathds{R}}
\newcommand{\CC}{\mathds{C}}
\newcommand{\I}{\mathds{I}}
\newcommand{\Q}{\mathds{Q}}
\newcommand{\X}{\mathds{X}}


\newcommand{\abs}[1]{\left\vert#1\right\vert}
\newcommand{\norm}[1]{\left\|#1\right\|}
\newcommand{\set}[1]{\left\{#1\right\}}
\newcommand{\seq}[1]{\left<#1\right>}
\newcommand{\co}[1]{\left[#1\right]}
\newcommand{\cc}[1]{\left(#1\right)}
