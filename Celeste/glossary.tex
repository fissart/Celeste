\newglossaryentry{asertividad}{name=Asertividad,description={Asertividad se refiere a la capacidad de expresar tus pensamientos, necesidades y sentimientos de manera clara, honesta y respetuosa, sin dañar a los demás ni permitir que te dañen a ti mismo. La asertividad es una habilidad importante tanto en las relaciones personales como en el ámbito laboral, ya que te permite comunicarte de manera efectiva y establecer límites saludables. La comunicación asertiva implica utilizar un lenguaje claro y directo, expresar tus sentimientos de manera honesta y respetuosa, y escuchar activamente a los demás. Al ser asertivo, puedes mejorar tus relaciones, aumentar tu autoconfianza y mejorar tu bienestar emocional.}}

\newglossaryentry{color}{name=Color,description={Es una propiedad visual que se percibe en los objetos debido a la luz que reflejan o emiten. Los colores son generados por la longitud de onda de la luz visible, que se compone de los colores del espectro (rojo, naranja, amarillo, verde, azul, índigo y violeta) y sus combinaciones. El color puede ser percibido de manera subjetiva y está presente en la naturaleza, el arte, la moda, la decoración, entre otros campos.}}

\newglossaryentry{comunicacion}{name=Comunicación,description={La comunicación es un proceso de intercambio de información entre dos o más personas o entidades. Se lleva a cabo a través de diferentes medios, como lenguaje verbal, escrito, gestual, entre otros, y tiene como objetivo transmitir conocimientos, sentimientos, ideas, mensajes, y establecer un vínculo entre los interlocutores. La comunicación es esencial en todo tipo de relaciones, ya sea a nivel personal, laboral o social. Es una herramienta fundamental para solucionar problemas, tomar decisiones y establecer acuerdos. Por tanto, es importante desarrollar habilidades de comunicación efectiva para poder relacionarnos de manera satisfactoria con las personas que nos rodean. La comunicación puede ser efectiva o inefectiva, dependiendo de la forma en que se desarrolla. Para que la comunicación sea efectiva es necesario que exista una retroalimentación adecuada, una comprensión mutua, respeto y empatía entre las partes involucradas.}}

\newglossaryentry{creatividad}{name=Creatividad,description={La creatividad es la capacidad de generar ideas originales y útiles que permiten resolver problemas o satisfacer necesidades de manera innovadora. También se refiere a la habilidad de crear algo nuevo a partir de elementos ya existentes. La creatividad no se limita a las artes y la cultura, sino que se extiende a todas las áreas de la vida, incluyendo la ciencia, la tecnología, la educación, los negocios y el desarrollo humano. La creatividad puede fomentarse a través de la exploración, la experimentación, la curiosidad y el pensamiento flexible.}}

\newglossaryentry{dibujo}{name=Dibujo,description={Es una representación gráfica de una imagen o idea, realizada con distintas técnicas, como lápiz, carboncillo, tinta, acuarela o técnicas digitales. Puede ser una creación artística o técnica, y se utiliza en diferentes ámbitos, como la arquitectura, el diseño gráfico, la ilustración, la ingeniería y las artes visuales. Los dibujos pueden ser abstractos o figurativos, y pueden transmitir emociones, sensaciones y pensamientos tanto del artista como del espectador.}}

\newglossaryentry{recrear}{name=Recrear,description={Volver a crear algo que ya existió en el pasado. Imitar o representar algo o alguien de manera cercana a la realidad. Divertirse o entretenerse realizando actividades recreativas.}}

\newglossaryentry{arte}{name=Arte,description={El arte es una manifestación cultural que se expresa a través de diversas formas y disciplinas artísticas, como la pintura, la escultura, la música, la literatura, la danza, el cine, el teatro, entre otras. A través del arte, los artistas pueden expresar emociones, sentimientos, ideas y reflexiones sobre la vida y la sociedad. El arte puede ser utilizado como una herramienta de comunicación para transmitir mensajes y provocar cambios sociales, políticos o culturales. Además, el arte tiene un impacto importante en la historia y la evolución de las distintas culturas y civilizaciones. El arte también puede ser una fuente de inspiración y entretenimiento para muchas personas. Puede tener un papel importante en la educación y el desarrollo de la creatividad de los individuos, tanto en términos personales como profesionales. En resumen, el arte es una expresión cultural muy valorada en todo el mundo por su capacidad de conmover, inspirar y transformar.}}

\newglossaryentry{caricatura}{name=Caricatura,description={Las caricaturas son una forma de arte que consiste en realizar una representación exagerada y humorística de una persona o situación. Se utilizan técnicas de dibujo para resaltar ciertas características físicas o de personalidad y se suelen utilizar colores vivos y formas exageradas.}}

\newglossaryentry{comico}{name=Cómico,description={El término "cómico" se refiere a alguien o algo que hace reír o provoca risa. Puede referirse a un humorista, comediante o actor que se dedica a hacer reír a la gente a través de su actuación o sus chistes. También puede referirse a una situación o historia divertida que tiene como objetivo principal provocar risa en el público. En general, un cómico es alguien que se caracteriza por su habilidad para generar humor y alegría en los demás.}}

\newglossaryentry{esbozo}{name=Esbozo,description={En español, la palabra "esbozo" se refiere a la acción de crear una representación inicial o preliminar de algo, ya sea una idea, un dibujo o un plan. Un esbozo puede ser el punto de partida para desarrollar una obra completa, como un boceto antes de hacer un dibujo detallado o un bosquejo antes de escribir un ensayo. También se puede utilizar en sentido figurado para referirse a una descripción general o resumida de algo.}}

\newglossaryentry{periodismo}{name=Periodismo,description={El periodismo es una forma de comunicación que consiste en recopilar, verificar y difundir información sobre eventos, acontecimientos y temas de interés público. Los periodistas son responsables de investigar, escribir y presentar noticias de manera objetiva y precisa.}}

\newglossaryentry{social}{name=Social,description={El término "social" puede tener diferentes significados dependiendo del contexto en el que se utilice. En general, se refiere a todo lo relacionado con la sociedad, las relaciones entre las personas y la interacción social.}}

\newglossaryentry{interrelacion}{name=Interrelación,description={La interrelación se refiere a la conexión o relación que existe entre diferentes elementos, fenómenos o individuos. Implica que dos o más cosas están conectadas, afectándose mutuamente o influyéndose entre sí. La interrelación puede ser de diversos tipos y puede darse en distintos ámbitos, como el social, económico, político, natural, etc.}}

\newglossaryentry{conceptos}{name=Conceptos,description={Una unidad cognitiva de significado. Nace como una idea abstracta (es una construcción mental) que permite comprender las experiencias surgidas a partir de la interacción con el entorno y que, finalmente, se verbaliza (se pone en palabras).}}



\newacronym{UNSCH}{UNSCH}{Uinversidad Nacional De San Cristobal De Huamanga}
%\newacronym{IFS}{IFS}{Iterated function system (sitemas de funciones iteradas)}

\newacronym{VI}{V1}{Variable independiente}
\newacronym{VD}{V2}{Variable dependiente}
\newacronym{V}{V}{Variable}

\newacronym{D}{D}{Dimensiones}

\newacronym{M}{M}{Malo}
\newacronym{B}{B}{Bueno}
\newacronym{R}{R}{Regular}
\newacronym{E}{E}{Excelente}
\newacronym{O}{O}{Observación}

\newacronym{IT}{IT}{$i$--ésimo items de los indicadores de la variable independiente}
\newacronym{ID}{ID}{Identificador de numeración}
\newacronym{Ci}{Ci}{$i$--ésimo items de la variable independiente}
\newacronym{Pi}{Pi}{$i$--ésimo items de la variable dependiente}

%%%%%%%%%%%%%%%%%%%%%%%%%%

%\glsxtrnewsymbol[description={Promedio de las observaciones correspondientes a la dimencion $i$ de la variable dependiente, donde $j=1,2,3,4,5$, $i=1,2,3$ son los promedios en cada observacion $O^i$ de las dimension $Di$ con el método tradicional y experimental respectivamente}]{w1}{$O^i_{\overline{x}_{Dj}^T}$ y $O^i_{\overline{x}_{Dj}^E}$}

\glsxtrnewsymbol[description={Sumatoria de las varianzas de cada elemento muestral}]{var}{$\sum s^2_i$}

\glsxtrnewsymbol[description={Varianzas de cada item}]{var2}{$\sum s^2_i$}

%\glsxtrnewsymbol[description={Promedio total}]{pi1}{$\overline{x}_T$}

%\glsxtrnewsymbol[description={Calificaciones de la variable dependiente en la experimentación tradicional y experimental respectivamente}]{pi2}{$\overline{V}^T$ y $\overline{V}^E$}

%\glsxtrnewsymbol[description={Frecuencia absoluta simple}]{f}{$f$}

%\glsxtrnewsymbol[description={Promedio de los items correspodientes a la dimensión de la variable dependiente con el método tradicional y experimental respectivamente}]{xD1}{$\overline{x}_{Di}^T$ y $\overline{x}_{Di}^E$}

%\glsxtrnewsymbol[description={Promedio de los items correspodientes a la variable dependiente	con el método tradicional y experimental respectivamente}]{xD2}{$\overline{x}_{T}^T$ y $\overline{x}_{T}^E$}