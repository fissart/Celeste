\documentclass[12pt,a4paper]{article}
\usepackage[spanish,es-tabla,es-noquoting]{babel}
\usepackage[utf8]{inputenc}
\usepackage[T1]{fontenc}
\usepackage{amsfonts}

\renewcommand{\rmdefault}{ptm}
\usepackage{mathptmx}
\DeclareSymbolFont{Xlargesymbols}{OMX}{cmex}{m}{n}
\DeclareMathSymbol{\Xsum}{\mathop}{Xlargesymbols}{80}

\DeclareSymbolFont{newtxletters}{OML}{ntxmi}{m}{it}
\SetSymbolFont{newtxletters}{bold}{OML}{ntxmi}{b}{it}
\DeclareMathSymbol{\mu}{\mathord}{newtxletters}{22}

\DeclareSymbolFont{Symbols}{OMS}{zplm}{m}{n}% Palatino
\DeclareMathSymbol{\infty}{\mathord}{Symbols}{"31}

\usepackage{lscape}
\usepackage{amsthm,tabls}
	\usepackage{midpage}
\usepackage{subfigure}
\usepackage{changepage}
\usepackage{longtable}
\usepackage{graphicx,xspace}
\usepackage[a4paper,left=3.54cm,right=2.54cm,top=2.54cm,bottom=2.54cm,centering]{geometry}
\usepackage{makecell}
\newcolumntype{M}[1]{>{\centering\arraybackslash}m{#1}}
\usepackage{multirow}
%\usepackage{showframe}
\usepackage{bigstrut}
\renewcommand\theadfont{\normalsize}
\usepackage{csquotes}
\usepackage{titlesec}
\usepackage{enumitem}
\usepackage{booktabs}
\usepackage{setspace}
\usepackage{multirow}
\usepackage{rotating}

\usepackage{fancyhdr}
\pagestyle{fancyplain}
\fancyhf{}
\fancyhead[R]{\thepage}
\renewcommand{\headrulewidth}{0pt}

\usepackage[apaciteclassic, numberedbib, nosectionbib, tocbib]{apacite}
\usepackage{usebib}
\bibinput{une}
\usepackage[tablewithout,belowskip=-15pt,aboveskip=0pt]{caption}
\renewcommand{\thefigure}{\arabic{figure}}
\renewcommand{\thetable}{\arabic{table}}
\usepackage{url}
\newcommand{\ce}{\centering}
\def\UrlBreaks{\do\.\do\@\do\\\do\/\do\!\do\%\do\7\do\D\do\_\do\\do\;\do\>\do\]%
  \do\)\do\,\do\?\do\'\do+\do\=\do\#}%
\def\UrlBigBreaks{\do\:\do@url@hyp}%
\def\UrlBreaks{\do\/\do-}
%\DefineBibliographyStrings{spanish}{andothers={et~al\adddot}}

\renewcommand{\thesection}{\Roman{section}}
\renewcommand{\thesubsection}{\arabic{section}.\arabic{subsection}}

\titleformat*{\section}{\normalsize\bfseries}
%\setcounter{secnumdepth}{5}
\newcommand\Tstrut{\rule{0pt}{2.35ex}}
\newcommand\Bstrut{\rule[-1.3ex]{0pt}{1.3ex}}
\newcommand{\TBstrut}{\Tstrut\Bstrut}

\renewcommand{\qedsymbol}{$\blacksquare$}

\newcommand{\ww}{0.015em}
\titlespacing*{\chapter}
{0pt} {5em plus \ww minus \ww} {0em plus \ww minus \ww}
\titlespacing*{\section}
{0pt} {0em plus \ww minus \ww}{0em plus \ww minus \ww}
\titlespacing*{\subsection}
{0pt} {0em plus \ww minus \ww}{0em plus \ww minus \ww} 
\titlespacing*{\subsubsection}
{0pt} {0em plus \ww minus \ww}{0em plus \ww minus \ww}
\titlespacing*{\paragraph} {0pt}{1.25ex plus 1ex minus .2ex}{2em}
\titlespacing*{\subparagraph} {\parindent}{3.25ex plus 1ex minus .2ex}{1em}
\titleformat*{\section}{\normalsize\bfseries}
\titleformat*{\subsection}{\normalsize\bfseries}
\titleformat*{\subsubsection}{\normalsize\bfseries}
\setlength{\parskip}{0.5em plus \ww minus \ww}

\renewcommand{\quote}{\list{}{\rightmargin=\leftmargin\topsep=0pt plus \ww minus \ww}\item\relax}
%%%%%%%%%%%%%%%%%%%%%%%%%%
\usepackage{xpatch}%space equation
\xapptocmd\normalsize{%
	\abovedisplayskip=0.5em plus \ww minus \ww
	\abovedisplayshortskip=0.5em plus \ww minus \ww
	\belowdisplayskip=0.5em plus \ww minus \ww
	\belowdisplayshortskip=0.5em plus \ww minus \ww
}{}{}
%%%%%%%%%%%%%%%%%%%%%
%\setlength\intextsep{0.5em}% plus \ww minus \ww}
\setlist{topsep=0em plus \ww minus \ww, parsep=0.5em plus \ww minus \ww, itemsep=0em} 

\setcounter{topnumber}{2}
\setcounter{bottomnumber}{2}
\setcounter{totalnumber}{4}
\renewcommand{\topfraction}{0.85}
\renewcommand{\bottomfraction}{0.85}
\renewcommand{\textfraction}{0.5}
\renewcommand{\floatpagefraction}{0.7}

\renewcommand{\baselinestretch}{1.2}
%%%%%%%%%%%% flex intertext
\makeatletter
\patchcmd\set@fontsize
{\f@linespread\baselineskip}
{\f@linespread\baselineskip  plus \ww minus \ww}
{}{}
\makeatother

\usepackage{tikz}

\DeclareCaptionLabelSeparator*{spaced}{\\[1.7ex plus .01em minus .01em]}
\captionsetup[table]{textfont=bf,format=plain,labelfont=bf,justification=justified,singlelinecheck=false,labelsep=spaced,skip=0pt}
\captionsetup[figure]{labelfont=bf,justification   = RaggedRight,singlelinecheck = off,skip=5pt,labelfont=it}
%\STsetdecimalsep{{.}}\spanishdecimal{.}

%\setlength\intextsep{0.5em}% plus \ww minus \ww}
\setlength{\textfloatsep}{0.0em plus \ww minus \ww}
%https://tex.stackexchange.com/questions/39017/how-to-influence-the-position-of-float-environments-like-figure-and-table-in-lat


\usepackage[acronym,symbols,nogroupskip]{glossaries-extra}
\usepackage{glossary-superragged}
\makeglossaries

\usepackage{dsfont}
%\usepackage{textcase}



\usepackage{textcase}
\newcommand{\autor}{Bach. Quispe Urbano, Luzmerly Celeste\xspace}
\newcommand{\asesor}{Lic. Canchari Soliz, Jose\xspace}

\newcommand{\expert}{\xspace}
\newcommand{\expertt}{\xspace}

\newcommand{\muestra}{20 estudiantes del colegio público \NoCaseChange{S}an \NoCaseChange{J}uán \NoCaseChange{A}yacucho\xspace}

\newcommand{\poblacion}{900 estudiantes del colegio público \NoCaseChange{S}an \NoCaseChange{J}uán \NoCaseChange{A}yacucho\xspace}

\newcommand{\lugar}{colegio publico \NoCaseChange{S}an \NoCaseChange{J}uan \NoCaseChange{A}yacucho\xspace}
%TITULO

\newcommand{\titulo}{LA CARICATURA PARA FORTALECER LA COMUNICACIÓN ASERTIVA EN ESTUDIANTES DE EDUCACIÓN SECUNDARIA DE HUAMANGA, AYACUCHO - 2023\xspace}
% VARIABLES
\newcommand{\variablei}{{{Caricatura}}\xspace}
\newcommand{\variabled}{{{Comunicación asertiva}}\xspace} 
% DIMENSIONES
% VI
\newcommand{\dimi}{{{Realidad}}\xspace}
\newcommand{\dimii}{{{Fantasía}}\xspace}
\newcommand{\dimiii}{{{Política}}\xspace}
\newcommand{\dimiiii}{{{Periodística}}\xspace}
% VD 
\newcommand{\dimd}{Empatía\xspace}
\newcommand{\dimdd}{Asertividad\xspace}
\newcommand{\dimddd}{Atención\xspace}
\newcommand{\dimdddd}{Pensamiento\xspace}
%%%%%%%%%%%%% Metodologia
\newcommand{\enfoque}{\emph{Cuantitativa}\xspace}
\newcommand{\tipo}{\emph{Pre-experimental}\xspace}
\newcommand{\nivel}{\emph{Explicativa}\xspace}
\newcommand{\diseno}{\emph{Experimental}\xspace}

%%%%%%%%%%% INDICADORES

% INDICADORES DE LA VARIABLE INDEPENDIENTE

% Inteligencia espacial
\newcommand{\gb}{Eventos en el hogar\xspace}
\newcommand{\gbb}{Eventos en el trabajo\xspace}
\newcommand{\gbbb}{Eventos en centros de estudios\xspace}
\newcommand{\gbbbb}{Eventos en centro de esparcimiento\xspace}

\newcommand{\gbbbbb}{Personajes ficticios\xspace}
\newcommand{\gbbbbbb}{Entornos ficticios\xspace}
\newcommand{\gbbbbbbb}{Situaciones absurdas\xspace}
\newcommand{\gbbbbbbbb}{Acciones atemporales\xspace}

\newcommand{\gbbbbbbbbb}{Eventos sociales\xspace}
\newcommand{\gbbbbbbbbbb}{Personajes importantes\xspace}
\newcommand{\gbbbbbbbbbbb}{Representación de protesta\xspace}
\newcommand{\gbbbbbbbbbbbb}{Opinión publica\xspace}

\newcommand{\gbbbbbbbbbbbbb}{Temas de actualidad\xspace}
\newcommand{\gbbbbbbbbbbbbbb}{Temas de controversia\xspace}
\newcommand{\gbbbbbbbbbbbbbbb}{Escándalos sociales\xspace}
\newcommand{\gbbbbbbbbbbbbbbbb}{Corrupción social\xspace}

% INDICADORES DE LA VARIABLE DEPENDIENTE

\newcommand{\fb}{Comprende al prójimo\xspace}
\newcommand{\fbb}{Entiende emociones\xspace}
\newcommand{\fbbb}{Atiende su entorno\xspace}
\newcommand{\fbbbb}{Compasión y apoyo\xspace}

\newcommand{\fbbbbb}{Expresa pensamientos claros\xspace}
\newcommand{\fbbbbbb}{Expresa opinión responsable\xspace}
\newcommand{\fbbbbbbb}{Realiza actos autónomos\xspace}
\newcommand{\fbbbbbbbb}{Resolución de conflictos\xspace}

\newcommand{\fbbbbbbbbb}{Enfoque\xspace}
\newcommand{\fbbbbbbbbbb}{Comprensión\xspace}
\newcommand{\fbbbbbbbbbbb}{Retención\xspace}
\newcommand{\fbbbbbbbbbbbb}{Aplicación\xspace}

\newcommand{\fbbbbbbbbbbbbb}{Procesa información\xspace}
\newcommand{\fbbbbbbbbbbbbbb}{Construye ideas\xspace}
\newcommand{\fbbbbbbbbbbbbbbb}{Formula juicios\xspace}
\newcommand{\fbbbbbbbbbbbbbbbb}{Realiza reflexiones\xspace}



% PROBLEMA
\newcommand{\problema}{¿En qué medida influye la caricatura para fortalecer la comunicación creativa en estudiantes del nivel secundario \NoCaseChange{A}yacucho 2023?\xspace}

\newcommand{\problemae}{¿En qué proporción influye la caricatura para fortalecer la empatía en la comunicación creativa?\xspace}

\newcommand{\problemaee}{¿De qué manera influye la caricatura para fortalecer la asertividad en la comunicación creativa?\xspace}

\newcommand{\problemaeee}{¿En qué proporción influye la caricatura para fortalecer la atención en la comunicación creativa?\xspace}

\newcommand{\problemaeeee}{¿En qué medida influye la caricatura para fortalecer el pensamiento en la comunicación creativa?.\xspace}



% OBJETIVO
\newcommand{\objetivo}{Determinar como la caricatura influye para fortalecer la comunicación creativa en estudiantes del nivel secundario \NoCaseChange{A}yacucho 2023.\xspace}

\newcommand{\objetivoe}{Determinar como la caricatura influye para fortalecer la empatía en la comunicación creativa\xspace}

\newcommand{\objetivoee}{Establecer como la caricatura influye para fortalecer la asertividad en la comunicación creativa\xspace}

\newcommand{\objetivoeee}{Precisar como la caricatura influye para fortalecer la atención en la comunicación creativa\xspace}

\newcommand{\objetivoeeee}{Determinar como la caricatura influye para fortalecer el pensamiento en la comunicación creativa\xspace}



% HIPOTESIS
\newcommand{\hipotesis}{Existe influencia significativa en la caricatura para fortalecer la comunicación creativa en estudiantes del nivel secundario Ayacucho 2023\xspace}

\newcommand{\hipotesise}{Se establece la influencia significativa en la caricatura para fortalecer la empatía en la comunicación creativa\xspace}

\newcommand{\hipotesisee}{Se corrobora la influencia significativa en la caricatura para fortalecer la asertividad en la comunicación creativa\xspace}

\newcommand{\hipotesiseee}{Se evidencia influencia significativa en la caricatura para fortalecer la atención en la comunicación creativa\xspace}

\newcommand{\hipotesiseeee}{Se establece influencia significativa en la caricatura para fortalecer el pensamiento en la comunicación creativa\xspace}




\newcommand{\N}{\mathds{N}}
\newcommand{\R}{\mathds{R}}
\newcommand{\CC}{\mathds{C}}
\newcommand{\I}{\mathds{I}}
\newcommand{\Q}{\mathds{Q}}
\newcommand{\X}{\mathds{X}}


\newcommand{\abs}[1]{\left\vert#1\right\vert}
\newcommand{\norm}[1]{\left\|#1\right\|}
\newcommand{\set}[1]{\left\{#1\right\}}
\newcommand{\seq}[1]{\left<#1\right>}
\newcommand{\co}[1]{\left[#1\right]}
\newcommand{\cc}[1]{\left(#1\right)}
\newglossaryentry{asertividad}{name=Asertividad,description={Asertividad se refiere a la capacidad de expresar tus pensamientos, necesidades y sentimientos de manera clara, honesta y respetuosa, sin dañar a los demás ni permitir que te dañen a ti mismo. La asertividad es una habilidad importante tanto en las relaciones personales como en el ámbito laboral, ya que te permite comunicarte de manera efectiva y establecer límites saludables. La comunicación asertiva implica utilizar un lenguaje claro y directo, expresar tus sentimientos de manera honesta y respetuosa, y escuchar activamente a los demás. Al ser asertivo, puedes mejorar tus relaciones, aumentar tu autoconfianza y mejorar tu bienestar emocional.}}

\newglossaryentry{color}{name=Color,description={Es una propiedad visual que se percibe en los objetos debido a la luz que reflejan o emiten. Los colores son generados por la longitud de onda de la luz visible, que se compone de los colores del espectro (rojo, naranja, amarillo, verde, azul, índigo y violeta) y sus combinaciones. El color puede ser percibido de manera subjetiva y está presente en la naturaleza, el arte, la moda, la decoración, entre otros campos.}}

\newglossaryentry{comunicacion}{name=Comunicación,description={La comunicación es un proceso de intercambio de información entre dos o más personas o entidades. Se lleva a cabo a través de diferentes medios, como lenguaje verbal, escrito, gestual, entre otros, y tiene como objetivo transmitir conocimientos, sentimientos, ideas, mensajes, y establecer un vínculo entre los interlocutores. La comunicación es esencial en todo tipo de relaciones, ya sea a nivel personal, laboral o social. Es una herramienta fundamental para solucionar problemas, tomar decisiones y establecer acuerdos. Por tanto, es importante desarrollar habilidades de comunicación efectiva para poder relacionarnos de manera satisfactoria con las personas que nos rodean. La comunicación puede ser efectiva o inefectiva, dependiendo de la forma en que se desarrolla. Para que la comunicación sea efectiva es necesario que exista una retroalimentación adecuada, una comprensión mutua, respeto y empatía entre las partes involucradas.}}

\newglossaryentry{creatividad}{name=Creatividad,description={La creatividad es la capacidad de generar ideas originales y útiles que permiten resolver problemas o satisfacer necesidades de manera innovadora. También se refiere a la habilidad de crear algo nuevo a partir de elementos ya existentes. La creatividad no se limita a las artes y la cultura, sino que se extiende a todas las áreas de la vida, incluyendo la ciencia, la tecnología, la educación, los negocios y el desarrollo humano. La creatividad puede fomentarse a través de la exploración, la experimentación, la curiosidad y el pensamiento flexible.}}

\newglossaryentry{dibujo}{name=Dibujo,description={Es una representación gráfica de una imagen o idea, realizada con distintas técnicas, como lápiz, carboncillo, tinta, acuarela o técnicas digitales. Puede ser una creación artística o técnica, y se utiliza en diferentes ámbitos, como la arquitectura, el diseño gráfico, la ilustración, la ingeniería y las artes visuales. Los dibujos pueden ser abstractos o figurativos, y pueden transmitir emociones, sensaciones y pensamientos tanto del artista como del espectador.}}

\newglossaryentry{recrear}{name=Recrear,description={Volver a crear algo que ya existió en el pasado. Imitar o representar algo o alguien de manera cercana a la realidad. Divertirse o entretenerse realizando actividades recreativas.}}

\newglossaryentry{arte}{name=Arte,description={El arte es una manifestación cultural que se expresa a través de diversas formas y disciplinas artísticas, como la pintura, la escultura, la música, la literatura, la danza, el cine, el teatro, entre otras. A través del arte, los artistas pueden expresar emociones, sentimientos, ideas y reflexiones sobre la vida y la sociedad. El arte puede ser utilizado como una herramienta de comunicación para transmitir mensajes y provocar cambios sociales, políticos o culturales. Además, el arte tiene un impacto importante en la historia y la evolución de las distintas culturas y civilizaciones. El arte también puede ser una fuente de inspiración y entretenimiento para muchas personas. Puede tener un papel importante en la educación y el desarrollo de la creatividad de los individuos, tanto en términos personales como profesionales. En resumen, el arte es una expresión cultural muy valorada en todo el mundo por su capacidad de conmover, inspirar y transformar.}}

\newglossaryentry{caricatura}{name=Caricatura,description={Las caricaturas son una forma de arte que consiste en realizar una representación exagerada y humorística de una persona o situación. Se utilizan técnicas de dibujo para resaltar ciertas características físicas o de personalidad y se suelen utilizar colores vivos y formas exageradas.}}

\newglossaryentry{comico}{name=Cómico,description={El término "cómico" se refiere a alguien o algo que hace reír o provoca risa. Puede referirse a un humorista, comediante o actor que se dedica a hacer reír a la gente a través de su actuación o sus chistes. También puede referirse a una situación o historia divertida que tiene como objetivo principal provocar risa en el público. En general, un cómico es alguien que se caracteriza por su habilidad para generar humor y alegría en los demás.}}

\newglossaryentry{esbozo}{name=Esbozo,description={En español, la palabra "esbozo" se refiere a la acción de crear una representación inicial o preliminar de algo, ya sea una idea, un dibujo o un plan. Un esbozo puede ser el punto de partida para desarrollar una obra completa, como un boceto antes de hacer un dibujo detallado o un bosquejo antes de escribir un ensayo. También se puede utilizar en sentido figurado para referirse a una descripción general o resumida de algo.}}

\newglossaryentry{periodismo}{name=Periodismo,description={El periodismo es una forma de comunicación que consiste en recopilar, verificar y difundir información sobre eventos, acontecimientos y temas de interés público. Los periodistas son responsables de investigar, escribir y presentar noticias de manera objetiva y precisa.}}

\newglossaryentry{social}{name=Social,description={El término "social" puede tener diferentes significados dependiendo del contexto en el que se utilice. En general, se refiere a todo lo relacionado con la sociedad, las relaciones entre las personas y la interacción social.}}

\newglossaryentry{interrelacion}{name=Interrelación,description={La interrelación se refiere a la conexión o relación que existe entre diferentes elementos, fenómenos o individuos. Implica que dos o más cosas están conectadas, afectándose mutuamente o influyéndose entre sí. La interrelación puede ser de diversos tipos y puede darse en distintos ámbitos, como el social, económico, político, natural, etc.}}

\newglossaryentry{conceptos}{name=Conceptos,description={Una unidad cognitiva de significado. Nace como una idea abstracta (es una construcción mental) que permite comprender las experiencias surgidas a partir de la interacción con el entorno y que, finalmente, se verbaliza (se pone en palabras).}}



\newacronym{UNSCH}{UNSCH}{Uinversidad Nacional De San Cristobal De Huamanga}
%\newacronym{IFS}{IFS}{Iterated function system (sitemas de funciones iteradas)}

\newacronym{VI}{V1}{Variable independiente}
\newacronym{VD}{V2}{Variable dependiente}
\newacronym{V}{V}{Variable}

\newacronym{D}{D}{Dimensiones}

\newacronym{M}{M}{Malo}
\newacronym{B}{B}{Bueno}
\newacronym{R}{R}{Regular}
\newacronym{E}{E}{Excelente}
\newacronym{O}{O}{Observación}

\newacronym{IT}{IT}{$i$--ésimo items de los indicadores de la variable independiente}
\newacronym{ID}{ID}{Identificador de numeración}
\newacronym{Ci}{Ci}{$i$--ésimo items de la variable independiente}
\newacronym{Pi}{Pi}{$i$--ésimo items de la variable dependiente}

%%%%%%%%%%%%%%%%%%%%%%%%%%

%\glsxtrnewsymbol[description={Promedio de las observaciones correspondientes a la dimencion $i$ de la variable dependiente, donde $j=1,2,3,4,5$, $i=1,2,3$ son los promedios en cada observacion $O^i$ de las dimension $Di$ con el método tradicional y experimental respectivamente}]{w1}{$O^i_{\overline{x}_{Dj}^T}$ y $O^i_{\overline{x}_{Dj}^E}$}

\glsxtrnewsymbol[description={Sumatoria de las varianzas de cada elemento muestral}]{var}{$\sum s^2_i$}

\glsxtrnewsymbol[description={Varianzas de cada item}]{var2}{$\sum s^2_i$}

%\glsxtrnewsymbol[description={Promedio total}]{pi1}{$\overline{x}_T$}

%\glsxtrnewsymbol[description={Calificaciones de la variable dependiente en la experimentación tradicional y experimental respectivamente}]{pi2}{$\overline{V}^T$ y $\overline{V}^E$}

%\glsxtrnewsymbol[description={Frecuencia absoluta simple}]{f}{$f$}

%\glsxtrnewsymbol[description={Promedio de los items correspodientes a la dimensión de la variable dependiente con el método tradicional y experimental respectivamente}]{xD1}{$\overline{x}_{Di}^T$ y $\overline{x}_{Di}^E$}

%\glsxtrnewsymbol[description={Promedio de los items correspodientes a la variable dependiente	con el método tradicional y experimental respectivamente}]{xD2}{$\overline{x}_{T}^T$ y $\overline{x}_{T}^E$}


\usepackage{catchfilebetweentags}
\makeatletter
\def\CatchFBT@sanitize{%
   \@sanitize
   \@makeother\{%
   \@makeother\}%
%   \endlinechar=`\^^J% <--- This line modified
}% \CatchFBT@sanitize
\makeatother



\glsaddall

%\usepackage{tocstyle}
%\usetocstyle{standard}

\usepackage{etoolbox}
\makeatletter
\pretocmd{\section}{\addtocontents{toc}{\protect\addvspace{-0.2em plus \ww minus \ww}}}{}{}
\pretocmd{\subsection}{\addtocontents{toc}{\protect\addvspace{0.5em plus \ww minus \ww}}}{}{}
\pretocmd{\subsubsection}{\addtocontents{toc}{\protect\addvspace{0.5em plus \ww minus \ww}}}{}{}
\makeatother

\usepackage{pgfplots}
\usetikzlibrary{intersections}
\pgfplotsset{compat=1.15}
\usepackage{amsmath,pgfplots,gincltex}
\begin{document}
%\SweaveOpts{concordance=TRUE}
\setlength{\abovecaptionskip}{0pt}
\setlength{\belowcaptionskip}{0pt}

\pagenumbering{roman}
\thispagestyle{empty}

	{\normalsize\centering\thispagestyle{empty} MINISTERIO DE EDUCACIÓN{}\\  SUPERINTENDENCIA NACIONAL DE EDUCACIÓN SUPERIOR UNIVERSITARIA (SUNEDU)\\ \vspace{0.65cm}
	
	{\large ESCUELA SUPERIOR DE FORMACIÓN ARTÍSTICA PÚBLICA\\ ``FELIPE GUAMÁN POMA DE AYALA''}\\  {\scriptsize CON NIVEL UNIVERSITARIO: LEY 30220}\\
	\vspace{0.65cm}  \includegraphics[height=6.5cm]{logo}\\
	\vspace{0.65cm}  {\large TÍTULO}\\ 
	\vspace{0.65cm}   
	\MakeUppercase{\Large\titulo}\\ 
	\vspace{0.65cm}   {\large PRESENTADO POR \\ 
		\vspace{0.65cm}  \autor}\\ 
	\vspace{0.65cm} \large TPROYECTO DE INVESTIGACIÓN PARA OPTAR EL TÍTULO PROFESIONAL DE LICENCIADA EN EDUCACIÓN ARTÍSTICA\\ 
	\vspace{0.3cm}  ESPECIALIDAD DE ARTES PLÁSTICAS\\ 
	\vspace{0.3cm}  AYACUCHO -- PERÚ\\ 
	\vfill
	{ 2023}
	
}

\newpage{}
\newpage

\flushbottom 

\renewcommand{\contentsname}{ÍNDICE DE CONTENIDOS}
%\addcontentsline{toc}{section}{Índice de contenidos}
\tableofcontents

\renewcommand\listfigurename{Lista de figuras} 
\addcontentsline{toc}{section}{Lista de figuras}
\listoffigures

\renewcommand\listtablename{Lista de tablas} 
\addcontentsline{toc}{section}{Lista de tablas}
\listoftables

\pagenumbering{arabic}
\setcounter{page}{1}

%\onehalfspacing
%\begin{center}
%\Large Anteproyecto de Tesis
%\end{center}
%\section{Título de la investigación}
%``\titulo''
%\tableofcontents
%\section{Datos generales}
%\subsection{Título} \titulo.
%\subsection{Tipo de investigación} Investigación aplicada
%\subsection{Autor} \autor
%\subsection{Especialidad} Matemática
%\subsection{Asesor} \asesor

\section{PLANTEAMIENTO DEL PROBLEMA}
\subsection{Determinación del problema}
\ExecuteMetaData[contenido.tex]{planteamiento}
\subsection{Formulación del problema}
\subsubsection{Problema General}
\problema.
\subsubsection{Problemas Específicos}
\begin{itemize}
\item \problemae.
\item \problemaee.
\item \problemaeee.
\item \problemaeeee.
\end{itemize}

%\subsection{Importancia y alcances de la investigación}
%\ExecuteMetaData[contenido.tex]{importancia}

\subsection{Justificación y viabilidad}
\ExecuteMetaData[contenido.tex]{justificacionconcepto}

\subsubsection{Justificación legal}
\ExecuteMetaData[contenido.tex]{justificacionlegal}
\subsubsection{Justificación científica}
\ExecuteMetaData[contenido.tex]{justificacioncientifica}
\subsubsection{Justificación social}
\ExecuteMetaData[contenido.tex]{justificacionsocial}
\subsubsection{Justificación metodológica}
\ExecuteMetaData[contenido.tex]{justificacionmet}
\subsubsection{Justificación práctica}
\ExecuteMetaData[contenido.tex]{justificacionpra}



%\subsection{Limitaciones de la investigación}
%\ExecuteMetaData[contenido.tex]{justificacionteo}
%\subsubsection{Limitación metodológica}
%\ExecuteMetaData[contenido.tex]{justificacionteo}
%\subsubsection{Limitación teorica}
%\ExecuteMetaData[contenido.tex]{justificacionteo}
%\subsubsection{Limitación de recurso}
%Los recursos económicos o presupuesto serán totalmente autofinanciados.
%\subsubsection{Limitación temporal}
%\ExecuteMetaData[contenido.tex]{justificacionteo}

\section{OBJECTIVOS}
Los objetivos de la investigación se refieren a las metas o propósitos que se pretenden alcanzar a través de la realización del estudio. Estos pueden variar según la naturaleza de la investigación y los intereses del investigador, pero algunos ejemplos comunes de objetivos de investigación incluyen:

\begin{enumerate}
\item  Identificar y describir un problema o fenómeno particular.
\item  Determinar las causas o factores que contribuyen a un problema o fenómeno.
\item  Explorar y analizar las opiniones, actitudes o comportamientos de un grupo de personas.
\item  Evaluar la eficacia o impacto de un programa, política o intervención.
\item  Desarrollar y probar una teoría o hipótesis.
\item  Generar conocimiento nuevo o contribuir a un área de estudio existente.
\item  Comparar y contrastar diferentes enfoques o intervenciones para determinar cuál es el más efectivo.
\item  Recopilar datos y obtener una comprensión profunda de un tema o área específica.
\item  Proponer recomendaciones o soluciones basadas en los hallazgos de la investigación.
\item  Contribuir al avance de la ciencia y el conocimiento en general.
\end{enumerate}

Estos son solo algunos ejemplos, y los objetivos de investigación pueden variar ampliamente dependiendo del tema y el contexto específico de la investigación. Es importante que los objetivos sean claros, específicos y alcanzables, lo que permitirá guiar el diseño, la ejecución y la interpretación de los resultados de la investigación. 

\subsubsection{Objetivo general}
\objetivo.
\subsubsection{Objetivos específicos}
\begin{itemize}
	\item \objetivoe.
	\item \objetivoee.
	\item \objetivoeee.
	\item \objetivoeeee.
\end{itemize}


\section{MARCO TEÓRICO}

\subsection{Antecedentes de estudio}
\subsubsection{Internacionales}
\ExecuteMetaData[contenido.tex]{internacional}
\subsubsection{Nacionales}
\ExecuteMetaData[contenido.tex]{nacional}
\subsubsection{Regionales}
\ExecuteMetaData[contenido.tex]{regional}

\subsection{Bases teóricas}
\subsubsection{Contexto de la investigación}
Ayacucho es una ciudad ubicada en los Andes de Perú, conocida por su historia, cultura y arte. Durante la época colonial, la ciudad fue un importante centro de producción artística, destacándose en áreas como la pintura, escultura, arquitectura y artesanía.

En Ayacucho se desarrolló la famosa escuela pictórica conocida como la "Escuela de Ayacucho", la cual produjo una gran cantidad de pinturas religiosas de estilo barroco. Estas obras se caracterizan por su gran detalle, colorido y dramatismo, y se encuentran en numerosas iglesias y conventos de la ciudad. \cite{velasco_historia_2008} 

En cuanto a la escultura, Ayacucho destaca por sus imágenes religiosas, talladas en madera o piedra. Estas esculturas son consideradas verdaderas obras de arte, con una gran expresividad y detalle en sus rasgos.

En el ámbito de la arquitectura, Ayacucho posee numerosos ejemplos de edificios coloniales, destacando la Plaza de Armas de la ciudad, rodeada de hermosas casonas coloniales y la Catedral de Ayacucho, construida en el siglo XVII.

\cite{velasco_historia_2008} menciona que en la artesanía, Ayacucho es famosa por sus tejidos, cerámicas y máscaras. Los tejidos son elaborados a mano por las comunidades indígenas de la región, utilizando técnicas tradicionales y materiales como la lana de alpaca y el algodón. Las cerámicas son elaboradas con arcilla y decoradas con motivos regionales, representando escenas de la vida cotidiana o eventos históricos. Las máscaras, por su parte, son utilizadas en festividades tradicionales y representan personajes folklóricos de la región.

En resumen, Ayacucho es una ciudad que posee una rica tradición artística, en la cual se fusionan influencias indígenas y españolas. El arte en Ayacucho ofrece una mirada única a la historia y la cultura de la región, y es un importante atractivo turístico para quienes visitan la ciudad.

\subsubsection{\variablei}%Representación de objetos matemáticos

Las caricaturas son una forma de arte que consiste en realizar una representación exagerada y humorística de una persona o situación. Se utilizan técnicas de dibujo para resaltar ciertas características físicas o de personalidad y se suelen utilizar colores vivos y formas exageradas.

Existen diferentes tipos de caricaturas, como las políticas, que se enfocan en realizar críticas o retratos humorísticos de políticos o líderes importantes. También están las caricaturas de celebridades, que se centran en resaltar las características físicas o de personalidad conocidas de figuras famosas \cite{gray_caricaturas_2015}.

Las caricaturas también están presentes en los cómics y en las tiras cómicas de los periódicos, donde se utilizan para contar historias humorísticas o transmitir mensajes de una manera divertida.

En la actualidad, las caricaturas también se han vuelto populares en la animación, donde se utilizan técnicas digitales para crear personajes y situaciones cómicas.

En resumen, las caricaturas son una forma de arte humorístico que utiliza técnicas de dibujo para resaltar características físicas o de personalidad de personas o situaciones. Son utilizadas en diferentes medios, como el arte, los cómics y la animación.

\noindent\textbf{\dimi}. De acuerdo a \cite{gray_caricaturas_2015}
son caricaturas que abordan temas relacionados con hechos reales o situaciones cotidianas comunes a los lectores, ya sea de manera simple o compleja. No todas las caricaturas de realidad son críticas, algunas tienen como único propósito hacer reír a los lectores al ridiculizar situaciones comunes, sin darles una mayor trascendencia. Se tiene los indicadores.
\begin{enumerate}
\item  Eventos en el hogar
\item  Eventos en el trabajo
\item  Eventos  en centros de estudios
\item  Eventos en centro de esparcimiento
\end{enumerate}




\noindent\textbf{\dimii}.	De acuerdo a \cite{gray_caricaturas_2015} este tipo de caricaturas abordan temas que no tienen relación alguna con los hechos de la realidad, pero no están completamente desconectadas de ella. Sus paisajes pueden ser poco usuales, ubicándose en lugares reales pero extraños o en dimensiones paralelas. Sin embargo, a través de situaciones absurdas, los personajes expresan sentimientos, emociones y pensamientos propios de la vida cotidiana. De esta manera, se ridiculizan o cuestionan formas habituales de sentir o pensar en la sociedad. La presencia de estos elementos permite al lector establecer una conexión y complicidad con el mensaje del autor. Se tiene los indicadores
\begin{enumerate}
	\item  Personajes ficticios
	\item Entornos ficticios
	\item  Situaciones absurdas
	\item  Acciones atemporales
\end{enumerate}

\noindent\textbf{\dimiiii}.	De acuerdo a \cite{grose_principios_2011}, la caricatura política tiene una larga tradición y ha evolucionado junto con la sociedad y la tecnología. En épocas antiguas, se realizaban a mano y se publicaban en periódicos y revistas, mientras que en la actualidad se han popularizado en las redes sociales y se realizan de forma digital.
Algunos de los temas más comunes que se abordan en la caricatura política son la corrupción, la injusticia, las desigualdades sociales y los abusos de poder. Los caricaturistas buscan generar reflexiones y críticas a través de la exageración de los rasgos físicos o emocionales de los personajes representados.
A lo largo de la historia, la caricatura política ha enfrentado censura y represión por parte de gobiernos y grupos poderosos. Sin embargo, su impacto en la sociedad ha sido innegable, ya que ha logrado transmitir mensajes de protesta y denuncia de manera efectiva y rápida.
En conclusión, la caricatura política es un medio de comunicación importante que utiliza la sátira y la exageración para abordar temas políticos y sociales. Su poder radica en su capacidad para generar críticas y reflexiones en la opinión pública, convirtiéndola en una herramienta esencial en la construcción de una sociedad informada y participativa. Se tiene los indicadores

\begin{enumerate}
\item  Eventos sociales
\item  Personajes importantes
\item  Representación de protesta
\item  Opinión publica
\end{enumerate}

\noindent\textbf{\dimiii}. Según \cite{gray_caricaturas_2015}	el propósito de la caricatura periodística es transmitir su mensaje de manera efectiva y provocadora, utilizando elementos visuales como la exageración de rasgos físicos y la representación simbólica de personas o ideas. A través de estas representaciones caricaturescas, se busca capturar la atención del lector y generar reflexión sobre la noticia o tema en cuestión.
Las caricaturas periodísticas suelen abordar temas de actualidad y controversia, como conflictos políticos, escándalos de corrupción, decisiones gubernamentales, problemas sociales o económicos, entre otros. A través de la sátira, el dibujante puede expresar críticas y opiniones de forma no literal, aprovechando la libertad creativa que le proporciona este género.
En los medios impresos, las caricaturas periodísticas suelen aparecer en la sección de opinión, compartiendo espacio con columnas y artículos de análisis. También se pueden encontrar en revistas y publicaciones especializadas. Con el auge de las redes sociales, las caricaturas periodísticas se han adaptado a nuevos medios, difundiéndose en plataformas digitales y alcanzando a un público más amplio.
En conclusión, la caricatura periodística combina el arte visual y el periodismo para transmitir mensajes críticos de manera humorística. Su objetivo es provocar reflexión y generar opinión sobre temas de interés público. Se tiene los indicadores


\begin{enumerate}
\item  Temas de actualidad
\item  Temas de controversia
\item  Escándalos sociales
\item  Corrupción social
\end{enumerate}


\subsubsection{\variabled}% Competencias del cálculo integral
La comunicación asertiva es un estilo de comunicación en el cual se expresa de manera clara, honesta y respetuosa. Se caracteriza por ser directa y enfocada en transmitir un mensaje de manera efectiva, sin herir o agredir a la otra persona.

Una comunicación asertiva implica expresar nuestras opiniones, necesidades y emociones de manera adecuada, sin ser agresivos, pasivos o manipulativos. Busca establecer una comunicación abierta y sincera, donde ambas partes se sientan escuchadas y respetadas.

Al utilizar una comunicación asertiva, podemos transmitir nuestros pensamientos y emociones de manera clara y libre de malentendidos, evitando conflictos innecesarios y fomentando la resolución pacífica de problemas.



\noindent\textbf{\dimd}. La empatía es la capacidad de ponerse en el lugar de otra persona y comprender cómo se siente. Es la habilidad de entender y responder a las emociones y los sentimientos de los demás, y se considera una habilidad social muy valiosa para crear relaciones saludables y efectivas. La empatía requiere una combinación de atención, comprensión y sensibilidad hacia los demás para poder conectarse con ellos y mostrarles compasión y apoyo.

\begin{enumerate}
\item  Comprende al prójimo
\item  Entiende emociones
\item  Atiende su entorno
\item  Compasión y apoyo
\end{enumerate}

\noindent\textbf{\dimdd}.	La asertividad es la capacidad de expresar nuestros pensamientos, opiniones, sentimientos y deseos de forma clara y directa, sin agredir o ser agredidos por los demás. Consiste en ser capaces de defender nuestros derechos y actuar de forma autónoma y responsable, sin caer en la sumisión o la agresividad. La asertividad es una habilidad importante en la comunicación interpersonal y en la resolución de conflictos.

\begin{enumerate}
\item Expresa pensamientos claros
\item Expresa opinión responsable
\item Realiza actos autónomos
\item Resolución de conflictos
\end{enumerate}
\noindent\textbf{\dimddd}.	La atención es un proceso cognitivo que nos permite enfocar nuestra mente en un estímulo específico o en una tarea determinada, al tiempo que ignoramos o filtramos otras distracciones. Es la capacidad de concentrarse y dirigir los recursos mentales hacia un objetivo o actividad en particular.
La atención es esencial para el aprendizaje, la memoria, la toma de decisiones y el rendimiento en general. Nos permite procesar la información de manera más eficiente, mejorar la comprensión, retención y aplicación de lo que aprendemos.
Existen diferentes tipos de atención, como la atención sostenida, que es la capacidad de mantener el enfoque en una tarea durante un período prolongado de tiempo; la atención selectiva, que implica seleccionar un estímulo relevante y filtrar las distracciones; y la atención dividida, que es la habilidad para atender a varias actividades o estímulos simultáneamente.
La atención puede verse afectada por diversos factores, como el cansancio, el estrés, la falta de motivación o la presencia de distracciones externas. Es posible mejorar y entrenar la atención a través de técnicas y estrategias como la meditación, la práctica de mindfulness, la organización y planificación, entre otras.

\begin{enumerate}
\item Enfoque
\item Comprensión
\item Retención
\item Aplicación
\end{enumerate}

\noindent\textbf{\dimdddd}. El pensamiento es la capacidad cognitiva y mental de los seres humanos para procesar información, construir ideas, formular juicios, realizar reflexiones y razonar. Es una actividad mental compleja que involucra diversos procesos como la percepción, la memoria, el lenguaje, la atención y la resolución de problemas. El pensamiento puede ser consciente o inconsciente, y se manifiesta a través de diferentes formas como el razonamiento lógico, la imaginación, la intuición, la creatividad y la toma de decisiones. Es la base de la actividad mental y de la construcción del conocimiento.

	\begin{enumerate}
\item Procesa información 
\item Construye ideas
\item Formula juicios
\item Realiza reflexiones
	\end{enumerate}
\setglossarysection{subsection}\flushbottom \printglossary[numberedsection,nonumberlist,title=Definiciones  de términos básicos]








\section{HIPÓTESIS}
\subsection{Hipótesis}
\subsubsection{Hipótesis principal}
      \hipotesis.
\subsubsection{Hipótesis específicas}
\begin{itemize}
\item \hipotesise.
\item \hipotesisee.
\item \hipotesiseee.
\item \hipotesiseeee.
\end{itemize}

\section{DEFINICIÓN CONCEPTUAL Y OPERACIONALIZACÍÓN DE VARIABLES }
%\subsubsection{Operacionalización de variables}
\begin{itemize}
  \item \textbf{Variable Independiente} (\variablei)
  \item \textbf{Variable Dependiente} (\variabled)
  \item \textbf{Variable Interviniente} (Capacidad artística, interrelación social)
\end{itemize}

\subsubsection{Definición conceptual de variables}

Las variables son los conceptos, atributos o propiedades de carácter social o natural, que tienden a variar, a cambiar, a modificar para asumir niveles o escalas. Se caracterizan por ser observables, medibles y cuantificables; por tal motivo asumen valores bien definidos.

\begin{itemize}
  \item \textbf{Variable Independiente (\variablei):} Una caricatura es un retrato que exagera o distorsiona la apariencia física de una o varias personas. Es en ocasiones un retrato de la sociedad reconocible, para crear un parecido fácilmente identificable y, generalmente, humorístico. También puede tratarse de alegorías.%Es la capacidad de generar imágenes \emph{dinámicas}, mental o gráficamente (sobre una superficie o una computadora) con el objetivo de estudiar un ente abstracto o figurativo lo cual se categorizó en dos dimensiones \MakeTextLowercase{\dimi} y \MakeTextLowercase{\dimii}.

  \item \textbf{Variable Dependiente (\variabled):} Nos podemos referir a cualquier tipo de comunicación, ya que el mero hecho de comunicar es un acto creativo, en tanto que, con las informaciones que tenemos añadimos algo más ya sea consciente o inconscientemente a la información que queremos transmitir.%Es el incremento de factores que identifican la eficacia del procesamiento mental matemático en un entorno social determinado, las que deben seguir una secuencia de pasos tales como la \MakeTextLowercase{\dimd}, \MakeTextLowercase{\dimdd}, \MakeTextLowercase{\dimddd} y \MakeTextLowercase{\dimdddd}.

  \item  \textbf{Variable Interviniente (Capacidad artística, interrelación social):} La capacidad artística se refiere a las habilidades, conocimientos y talentos relacionados con la creación y expresión artística. Esto incluye habilidades en áreas como la pintura, la música, la danza, la escritura, el teatro y más.
  
  La interrelación social se refiere a la capacidad de interactuar y relacionarse con otras personas de manera efectiva y satisfactoria. Implica habilidades como la comunicación, la empatía, la escucha activa y la capacidad de trabajar en equipo.
  
  La capacidad artística puede tener un impacto positivo en la interrelación social. Participar en actividades artísticas puede ayudar a mejorar las habilidades de comunicación y expresión, fomentar la creatividad y promover la interacción con otras personas que comparten intereses similares.
  
  Además, las actividades artísticas pueden proporcionar un espacio seguro para la expresión emocional y personal, lo que a su vez puede ayudar a mejorar el bienestar emocional y la confianza en uno mismo. Esto puede facilitar la interrelación social al brindar a las personas una mayor seguridad y comodidad al interactuar con los demás.
  
  La interrelación social, a su vez, puede beneficiar la capacidad artística. Al interactuar con otras personas y compartir ideas y perspectivas, los artistas pueden ampliar su horizonte artístico y encontrar nuevas inspiraciones y enfoques. Además, la retroalimentación y el apoyo de la comunidad artística pueden ayudar a desarrollar y mejorar las habilidades artísticas.
  
  En resumen, la capacidad artística y la interrelación social están estrechamente relacionadas y se refuerzan mutuamente. Participar en actividades artísticas puede mejorar las habilidades de interrelación social, al tiempo que las habilidades sociales pueden beneficiar la creatividad y el desarrollo artístico.
  
  % La primera influye en poca tendencia hacia la lógica y los números y desinterés hacia los factores de productividad matemática, la segunda debido a la escasa llegada hacia la información idónea y adecuada asistida.

\begin{displayquote}
\emph{A veces llamada la variable de confusión, vincula las variables independientes y dependientes. En ciertas situaciones, la relación entre una variable independiente y una variable dependiente no puede establecerse sin la intervención de otra variable. La causa, o variable independiente, tendrá el efecto supuesto solo en presencia de una variable intermedia.} Grinnell (1988, p. 203 citado en \citeauthor{kumar}, p.~75)
\end{displayquote}

\end{itemize}

\subsubsection{Operacionalización de variables}
\ExecuteMetaData[contenido.tex]{ov}








\section{ASPECTO METODOLÓGICO}

\subsection{Enfoque de la investigación}
Investigación de enfoque cuantitativo debido a que se manipulan datos numéricos con el objetivo de deducir la veracidad de las hipótesis descritas en esta investigación mediante el uso adecuado de las estadísticas descriptiva e inferencial.

De acuerdo a \cite{bernal} La orientación cuantitativa se fundamenta en el cálculo de características de los fenómenos sociales, lo cual presupone derivar de un cuadro conceptual adecuado al problema examinado, una cadena de proposiciones que enuncien relaciones entre si las variables experimentadas de forma justificada. Este procedimiento tiende a extender y sistematizar resultados. Según \cite{inv1}.

\begin{displayquote}
{La orientación cuantitativa se constituye, de un conjunto de términos que es secuencial y demostrativo. Cada fase antecede a la subsiguiente y no es posible omitir o evitar sucesos. La disposición es inflexible. Parte de una idea que va delimitándose y, una vez definida, se proceden a generar objetivos y cuestiones de exploración, se revisa la bibliografía y se elabora un marco o una configuración teórica. De las cuestiones se forman hipótesis y establecen variables; se bosqueja un procedimiento para experimentar la cual se conoce como diseño, se calculan las variables en un determinado contexto; se exploran los datos obtenidos, recurriendo a procesos estadísticos, y se deduce una serie de conclusiones con relación a la o las hipótesis}. (p. 97)
\end{displayquote}

\subsection{Método de investigación}

La investigación es de tipo aplicada pues esta investigación busca una teoría didáctica que pueda ser aplicada en el campo educativo una vez sea probada la veracidad y objetividad de las hipótesis planteadas. 

\subsection{Tipo de investigación}
La investigación es de tipo aplicada, que tiene por finalidad contribuir al conocimiento científicos de aprendizaje y herramientas pedagógicas. De acuerdo a textcite{santiago}.%\citeauthor{Duval} (1999, p. 15 citado en \citeauthor{ines}, \citeyear{ines}, p.~15).

\begin{displayquote}
\emph{Es también llamada práctica, empírica, activa y se encuentra íntimamente ligada a la investigación básica, ya que depende de sus descubrimientos y aportes teóricos para poder generar beneficios y bienestar a la sociedad. Se fundamenta en la investigación teórica; su finalidad especifica es aplicar las teorías existentes a la producción de normas y procedimientos tecnológicos para controlar situaciones  o procesos de la realidad.} (p.~15)
\end{displayquote}

Al respecto cite{cerezal}, describe la investigación aplicada, como el manejo de las sapiencias, en la experiencia cotidiana, para emplearlos, en la totalidad de los casos, en beneficio de la humanidad.

SALDAÑA, J. (1998). ``Se refiere a un estudio de investigación en
el que se manipulan deliberadamente una o más variables independientes
(supuestas causas)
para analizar las consecuencias de esa manipulación sobre una o más
variables dependientes
(supuestos efectos), dentro de una situación de control para el investigador''.

%\subsubsection{Diseño de investigación}
%La investigación es de nivel explicativa experimental, porque se busca la causa y efecto en cada una de las variables de estudio, con manipulación de la variable independiente, es decir, se manipula la variable \MakeTextLowercase{\variablei}, para efectos o influencias en la \MakeTextLowercase{\variabled}.

%Por lo que \cite{khotari} explica que en una investigación experimental, de pruebas de hipótesis, cuando un grupo está expuesto a las condiciones habituales, se denomina \emph{grupo de control} (A), pero cuando lo esta en alguna condición nueva o especial, se denomina un \emph{grupo experimental } (B). Si los grupos A y B están expuestos a programas de estudios especiales, entonces ambos grupos se denominarían grupos experimentales. Es posible diseñar estudios que incluyan sólo grupos experimentales o estudios experimentales que incluyen tanto grupos experimentales como de control.

%In an experimental hypothesis-testing research when a
%group is exposed to usual conditions, it is termed a ‘control group’, but when the group is exposed to
%some novel or special condition, it is termed an ‘experimental group’. In the above illustration, the
%Group A can be called a control group and the Group B an experimental group. If both groups A and
%B are exposed to special studies programmes, then both groups would be termed ‘experimental
%groups.’ It is possible to design studies which include only experimental groups or studies which
%include both experimental and control groups.

\subsection{Diseño de investigación}
El diseño de investigación \MakeTextLowercase{\diseno}, que se estructura de acuerdo a la Figura \ref{figg}.
%individuos de la población iguales oportunidades de ser seleccionados, porque los grupos ya están formados con los estudiantes de la serie 200 en la \lugar.

\begin{figure}[ht!]\centering
	\begin{tikzpicture}[> = stealth,shorten > = 2pt,semithick]
	\node[] at (0,0)  (T){$M$};
	\node[] at (2,1.3)  (o1){$O_1$};
	\node[] at (5,1.3)  (w){$V_1$};
	\node[] at (2,0)  (r){$r$};
	\node[] at (2,-1.3)  (o2){$O_2$};
	\node[] at (5,-1.3)  (ww){$V_2$};
	\path[->] (T) edge node {} (o1);
	\path[->] (T) edge node {} (o2);
	\path[<-] (o1) edge node {} (w);
	\path[<-] (o2) edge node {} (ww);
	\path[->] (o1) edge node {} (r);
	\path[->] (o2) edge node {} (r);
	\end{tikzpicture}
	\caption{Diseño \MakeTextLowercase{\diseno}}
	\label{figg}
\end{figure}

Donde $M$ es la muestra, $O_1$ es la variable 1, \MakeTextLowercase{\variablei},
$O_2$ es la variable 2, \MakeTextLowercase{\variabled} y
$r$ es la relación entre variable 1 y variable 2.

El diseño de la investigación es cuasi -- experimental de un mismo grupo de trabajo con pre y pos -- prueba, porque se tomará un solo grupo experimental. De acuerdo a textcite{design} este diseño permitirá aplicar los módulos de experimentación en un periodo determinado y luego se trabajará de manera tradicional, alternando sucesivamente. Cuyo esquema de referencia es:

De acuerdo a :
\begin{displayquote}
\emph{Al elegir un diseño se tiene en cuenta una secuencia de procesos para aplicarlos en un determinado problema y deducir resultados que verifiquen las hipótesis del problema. En este caso se tendrá el diseño propuesto con un solo grupo, que permitirá aplicar la primera variable sobre la segunda, para fortalecer la elaboración plástica tridimensional.}. \cite{kothari_research_2004}[p 243]
\end{displayquote}



\subsection{Población y muestra}

\subsubsection{Población}

Constituida por \poblacion, matriculados en el semestre impar del año 2019. %Los criterios que se considerarán en la selección de la muestra se describen en el Cuadro \ref{apt7}


\begin{table}[ht!]
\caption{Criterio de inclusión y exclusión}\label{apt7}
%\centering
\begin{tabular}{ccc}\Xhline{2pt}
Población&Aptos&No aptos\\\midrule
\makecell*[{{p{3.5cm}}}]{\centering Estudiantes matriculados del semestre impar} &
\makecell*[{{p{3.9cm}}}]{\centering Estudiantes regulares\\Asistentes puntuales} &\makecell[c]{ Estudiantes repitentes \\
   Estudiantes retirados \\
   Estudiantes del quinto superior\\
   Estudiantes no asistentes}\\\bottomrule
\end{tabular}
\end{table}


De acuerdo a \cite{inv1}, la población o universo es el agregado de todos los casos que concuerdan con determinadas descripciones o características, es decir es un conjunto de elementos que tiene al menos una misma característica que pueden ser medidos cuantitativamente o cualitativamente.

\subsubsection{Muestra}

En la presente investigación la muestra será no probabilística e intencional compuesta por un solo grupo experimental de \muestra, al cual se le aplicará el diseño \MakeTextLowercase{\diseno} de un mismo grupo con dos tipos de pruebas en series temporales equivalentes (Alternado).

Según \cite{hernandez_sampieri_metodologiinvestigacion_2014}, la muestra es una subcolección de elementos de la población seleccionada, de donde se obtendrá información para el desarrollo del estudio y del cual se hará medicines y observaciones de la variable dependiente e independiente, interrelacionada.

\subsubsection{Muestreo}

El muestreo será no probabilístico e intencional. Esta técnica consiste  en obtener una muestra donde los elementos  se recogen de manera aleatoria.

\cite{alzina_metodologiinvestigacion_2004} se refiere a la muestra no probabilística como aquella donde las unidades de la población tiene la misma posibilidad de ser elegidas y pertenecer a la muestra. Es deliberado, ya que esta práctica se opera en poblaciones indistintas. Aquí el especialista, conociendo la población y con buen juicio resuelve que características de análisis compondrá la muestra.


\subsection{Técnicas e instrumentos de recolección de datos}

\subsubsection{Técnica de recolección de datos}

\paragraph{Observación} Para el análisis de la influencia de la variable independiente sobre la variable dependiente se utilizó las técnicas de experimentación y observación, empleándose como instrumentos el plan de experimentación y la ficha de observación. Esta técnica que permitió recoger datos de la influencia de la variable independiente \MakeTextLowercase{\variablei} sobre la variable dependiente \MakeTextLowercase{\variabled}, en el proceso de experimentación.


%
Según \cite{paz_metodologiinvestigacion_2014} el método de observación es el método más utilizado, especialmente en estudios relacionados con ciencias. La observación se convierte en una herramienta científica y en el método de recolección de datos para el investigador, cuando sirve a un propósito de investigación formulado, se planifica y registra sistemáticamente y se somete a controles de validez y fiabilidad.

\paragraph{Prueba pedagógica} Técnica que permitió recoger de manera escrita del  \MakeTextLowercase{\variablei} logro del aprendizaje de la variable dependiente \MakeTextLowercase{\variabled}, en el proceso de la experimentación

\cite{hernandez_sampieri_metodologiinvestigacion_2014} refiere: ``Es un proceso a través del cual se compara una unidad preestablecida y que la evaluación es un proceso que consiste en obtener infom1ación sistemática y objetiva acerca de un fenómeno e interpretar dicha información a fin de seleccionar entre distintas alternativas de decisión'' (p.~68).

\paragraph{Experimental} Técnica que permitió aplicar los módulos de experimentación de la variable \variablei.


{Según \cite{khotari} el método de observación es el método más utilizado, especialmente en estudios relacionados con ciencias. La observación se convierte en una herramienta científica y en el método de recolección de datos para el investigador, cuando sirve a un propósito de investigación formulado, se planifica y registra sistemáticamente y se somete a controles de validez y fiabilidad.}

\subsubsection{Instrumento de recolección de datos}

\paragraph{Plan de experimentación.} Documento pedagógico que contiene el proceso de aplicación de las estrategias pictóricas (ver anexo).

\paragraph{La ficha de observación} En el cual se incluyen indicadores que permitieron conocer  los indicadores de la variable independiente planteadas que influirán en el desarrollo de los indicadores de la variable dependiente en la muestra. En este caso el proceso será observado a través del evento práctico teórico que hicieron los observados.



%La ficha de observación por \cite{bernal}, es la exploración visual de lo que sucede en un contexto existente, en un fenómeno explícito, especificando y estableciendo los hechos acertados de acuerdo con algún diseño previsto, manipula un herramienta o prontuario impreso, reservado a conseguir respuestas sobre el problema en estudio.
%
%
%
\paragraph{La ficha de opinión}
%
Este instrumento permitirá recoger la opinion de los estudiantes con respecto al uso de la \MakeTextLowercase{\variablei}, y averiguar el impacto de esta. De acuerdo a cite{opinion}. Una ficha de opinión, es una ficha en la que personas externas a la investigación escribe plasma lo que piensa en relación al texto o estudio que se está realizando.
%
%
%
Esta ficha puede tener: Número (es para tener orden de las fichas), Autor (la persona quien lo escribe), Tema de que se trata, Items o cuestionarios y Nota (en caso requerido).
%
%
%
%%URL del artículo: %http://www.ejemplode.com/13-ciencia/2411-ejemplo_de_ficha_de_opinion.html
%%Leer completo: ejemplos de Ficha de opinión
%%Los indicadores que conformaran la estructura de la ficha de
%%observación fueron elaborados a partir de la operacionalización
%%de la variable independiente, siendo las posibilidades de sus
%%respuestas: No, A veces y Sí.
\paragraph{Prueba escrita} Instrumento que permitirá recoger datos del desarrollo de las capacidades matemáticas  \MakeTextLowercase{\dimd},  \MakeTextLowercase{\dimdd}, \MakeTextLowercase{\dimddd} y \MakeTextLowercase{\dimddd}; instrumento elaborado de acuerdo a los indicadores descritos en el presente trabajo de investigación.

Las competencias y sus valoraciones cualitativas y cuantitativas se describen en el siguiente Cuadro \ref{evaluacion}.

\begin{table}[ht!]
%\centering
\caption{Criterio de calificación}\label{evaluacion}
\begin{tabular}{cccc}\Xhline{2pt}
\bf \multirow{2}*[-.01cm]{Capacidades}&\bf Valoración&\bf Valoración\\
&\bf  cualificada&\bf  cuantificada\\\midrule
\multirow{5}*[-.01cm]{\makecell*[{{p{7cm}}}]{\centering  \dimd  \dimdd \dimddd \dimddd}}&Excelente&17 -- 20\\
	&Bueno&13 -- 16\\	
	&Regular&09 -- 12\\	
	&Malo&05 -- 08\\
	&Deficiente&00 -- 04\\	
\bottomrule
\end{tabular}
\end{table}

%%  Es la forma de justipreciar las destrezas, conductas, contenidos, procurando un visto bueno, puntaje, calificativos o un concepto, este actúa como un dispositivo de revisión durante el proceso de enseñanza de ciertos indicadores establecidos y la exploración del efecto.
%%Se emplea, generalmente, para observar la conducta de los observados.
%%
%%
%%
%%Por \cite{inv1}, enuncia. Es un instrumento de medición cuyo propósito es que el estudiante demuestre la adquisición de un aprendizaje cognoscitivo, el dominio de una destreza o el desarrollo progresivo de una habilidad. Por su naturaleza, requiere respuesta escrita por parte del estudiante.
%
%\paragraph{Módulos de experimentación} Módulos que permitirán elaborar los materiales de aplicación de la variable \MakeTextLowercase{\variablei} con el objetivo de generar el \MakeTextLowercase{\variabled}. El plan experimentación se muestra en el Anexo 7.
%
%
%
%
%%Los indicadores que conforman la estructura de la Lista de cotejo
%%se obtuvieron a partir de la operacionalización de la variable
%%dependiente, siendo las posibilidades de sus respuestas: No, A veces y Sí.
%%Aplicado por los investigadores en las sesiones de aprendizajes y
%%con los resultados.
%
%

\subsection{Tratamiento estadístico de los datos}
Con el objetivo de procesar adecuadamente los datos y sacar conclusiones confiables, se utiliza  las siguientes estadísticas.
 
\subsubsection{La estadística descriptiva} Esta estadística se utiliza en el estudio  del comportamiento de los datos, para inferir conclusiones relativas de nuestras variables y la relación entre ellas.  
\begin{enumerate}
\item Tendencia central (media, mediana y moda),
\item dispersión o variabilidad (desviación estándar, la varianza y la regresión estándar),
\item curtosis (leptocurtica, mesocúrtica y platicúrtica) y 
\item asimetría (asimétrica positiva, simétrica y asimétrica negativa).
\end{enumerate}

\subsubsection{La estadística inferencial} Esta estadística se utilizará para probar, la normalidad y la homocedasticidad  de los datos, además de las prueba de hipótesis.

Cabe mencionar que se hará uso del software R Sweave y Excel para el procesamiento de los datos.

%\subsubsection{Validación de instrumentos}
%Consistirá en validar el contenido del instrumento por el criterio de jueces o expertos (con maestría o doctorado), quienes verificarán y evaluarán la coherencia y secuencialidad de los instrumentos.
%
%
%
%La opinión de los expertos consultados permitirá establecer la validez de los instrumentos que se empleará en la investigación. La validez se establecerá mediante el método del juicio de expertos.
%
%
%
%\begin{adjustwidth}{1cm}{}\gggl
%\emph{La validez, en términos generales, se refiere al grado en que un instrumento mide
%realmente la variable que pretende medir. Por ejemplo, un instrumento válido para medir la inteligencia debe medir la inteligencia y no la memoria. Un método para medir
%el rendimiento bursátil tiene que medir precisamente esto y no la imagen de una empresa. Un ejemplo aunque muy obvio de completa invalidez sería intentar medir el peso de los objetos con una cinta métrica en lugar de con una báscula.} \cite[p.~200]{inv1}
%\end{adjustwidth}

\subsection{Procedimientos}
\subsection{Estadísticos descriptivos}
\subsection{Estadísticos inferenciales}

Existen  diversos procedimientos para calcular la confiabilidad de un instrumento de medición. Todos utilizan formulas que producen coeficientes de confiabilidad. La mayoría de estos coeficientes pueden oscilar entre cero y uno donde un coeficiente cero significa nula confiabilidad y uno representa  un máximo de confiabilidad.

La confiabilidad de consistencia interna del instrumento, se determinará con la prueba piloto, en una muestra de 10 estudiantes que no serán miembros de la muestra, aplicando mitades de Coeficiente de Pearson y la corrección de Spearman Brow, la fórmula referencial es la siguiente: $$r_{xy}=\frac{n\Xsum xy-\cc{\Xsum x}\cc{\Xsum y}}{\sqrt{n\Xsum x^2-\cc{\Xsum x}^2}\sqrt{n\Xsum y^2-\cc{\Xsum y}^2}}$$

La corrección con Spearman Brow $$r=\frac{2r_{xy}}{1+r_{xy}}$$

	Donde:
$n$ tamaño de muestra
$x$ es la puntuación de la primera mitad (Impares)
$y$ es la puntuación de la segunda mitad (Pares)
$r_{xy}$ es el coeficiente Pearson
$r$ es el coeficiente de Spearman Brow. Si $0\leq r\leq 0.8$ se considera no confiable y si $0.8< r \leq1$ se considera confiable.

\begin{displayquote}
\emph{La confiabilidad de un instrumento de medición es el grado en que su aplicación
repetida al mismo individuo u objeto genera resultados similares. Por ejemplo, si se midiera en este momento la temperatura ambiental usando un termómetro y éste indicara que hay $22^\circ C$, y un minuto más tarde se consultara otra vez y señalara $5^\circ C$, tres minutos después se observara nuevamente y éste indicara $40^\circ C$, dicho termómetro no sería confiable, ya que su aplicación repetida produce resultados distintos. Asimismo, si una prueba de inteligencia (Intelligence Quotient, IQ) se aplica hoy a un grupo de personas y da ciertos valores de inteligencia, se aplica un mes después y proporciona valores diferentes, al igual que en subsecuentes mediciones, tal prueba no sería confiable. Los resultados no son coherentes, pues no se puede confiar en ellos.
} \cite[p.~61]{inv1}\end{displayquote}

\begin{table}[ht!]
% \centering\STautoround{3}
{%\tiny
\caption{Prueba de confiabilidad de la ficha de observación}\label{pret:1}
%\vspace{0.5cm}
\begin{tabular}{cccccccccccc}\Xhline{2pt}
AI&\multicolumn{10}{c}{INDICADORES}&$\sum$\\\midrule
&IN1&IN2&IN3&IN4&IN5&IN6&IN7&IN8&IN9&IN10&\\\cline{2-11}
1 &&&&&&&&&&&\\
2 &&&&&&&&&&&\\
3 &&&&&&&&&&&\\
4 &&&&&&&&&&&\\
5 &&&&&&&&&&&\\\midrule
$\sum$         &&&&&&&&&&&\\
$\overline{x}$ &&&&&&&&&&&\\
$s_i^2$        &&&&&&&&&&&\\
$s_i$          &&&&&&&&&&&\\\bottomrule
\end{tabular}\\\vspace{0.5cm}
{\normalsize Fuente: Datos obtenidos de la muestra}
}
\end{table}


El pre test con los datos obtenidos mediante la calificación de los expertos en artes plásticas la ficha de observación generó ... de acuerdo a la Tabla ref{pret}, el instrumento (la ficha de observación)es confiable pues $0.8<\alpha<1$.

$$\sum s_i^2$$
$$s_t^2$$
$$\alpha=\frac{k}{k-1}\left(1-\frac{\sum s_i^2}{s_t^2}\right)$$


\section{ASPECTO  ADMINISTRATIVO}
\subsection{Recursos}

\subsubsection{Recursos humanos}
\begin{enumerate}
  \item Responsable de la investigación
  \item Profesor Asesor
  \item Estudiantes de la \lugar.
  \item Autoridades de la EPG de la UNE “EGV” La Cantuta.
  \item Autoridades de la Universidad Nacional De San Cristóbal De Huamanga.
\end{enumerate}

\subsubsection{Recursos materiales}
\begin{enumerate}
\item Bibliografía especializada e Internet %\cite{geer}
%\item Materiales y equipos de laboratorio %\cite{ctan}
\item Materiales de escritorio y computadora %\cite{rahtz:96-1}
\item Otros materiales %\cite{girou:01:}
\end{enumerate}

\subsubsection{Recursos financieros}
\begin{enumerate}
\item Auditorio de la \lugar
\item Laptop y Data Show
\end{enumerate}

\subsection{Presupuesto}
Materiales y equipos (financiados) refiérase al Cuadro \ref{presup}.

\begin{table}[ht!]
\caption{Presupuesto}\label{presup}
\begin{tabular}{ccc}\toprule
$\text N^\circ$&  Materiales y servicios &\multicolumn1c{Gastos en S/} \\\midrule
1 & Materiales didácticos       & 135.30               \\
2 & Impresión                             & 102.50             \\
3 & Libros                                & 300.90            \\
4 & Empastados de la tesis (4 unidades)   & 80.10              \\
5 & Gastos de movilidad                   & 90.00                \\
6 & Otros gastos                          & 747.00               \\\midrule
&TOTAL                                   &     wwwww\\
\bottomrule
\end{tabular}
\end{table}

\subsection{Cronograma de actividades}
Temporalización-cronograma de acciones refiérase al Cuadro \ref{cronog}.
\begin{table}[ht!]

\caption{Cronograma de acciones}\label{cronog}
\begin{tabular}{cp{6cm}ccccccccccccc}
\toprule
\ce $\text N^\circ$&Actividades&\multicolumn{11}{c}{2019}\\%&\multicolumn{2}{c}{2019}\\
\midrule
&&E&F&M&A&M&J&J&A&S&O\\\cline{3-13}
1&Análisis bibliográfico&$\bullet$&$\bullet$&&&&&&&&\\
2&Elaboración del marco teórico&$\bullet$&$\bullet$&&&&&&&&\\
3&Elaboración de los instrumentos&&$\bullet$&$\bullet$&&&&&&&\\
4&Prueba de los instrumentos&&&$\bullet$&&&&&&&\\
5&Aprobación del proyecto&&&&$\bullet$&$\bullet$&$\bullet$&&&&\\
6&Recolección de datos&&&&&&&$\bullet$&$\bullet$&&\\
7&Procesamiento y análisis de datos&&&&&&&&$\bullet$&&\\
8&Redacción de la tesis&&&&&&&&&$\bullet$&&\\
9&Presentación de la tesis&&&&&&&&&$\bullet$&\\
10&Sustentación&&&&&&&&&&$\bullet$\\
\bottomrule
\end{tabular}
\end{table}

%\begin{table}[ht!]
%\centering
%\begin{spreadtab}{{tabular}{clc}}\Xhline{2pt}
%@ RECURSOS & @ MATERIALES Y SERVICIOS &@\multicolumn1c{GASTOS EN S/} \\\midrule
%@\multirow{4}*[0cm]{Materiales} & @Materiales para la escultura          @& 135.30               \\
%@& @Impresiones                           & 102.50             \\
%@& @Libros                                & 300.90            \\
%@& @Libros                                & 300.90            \\\midrule
%@\multirow{4}*[0cm]{Servicios} & @Empastados de la tesis (4 unidades)   & 80.10              \\
%@ & @Gastos de movilidad                   & 90.00                \\
%@ & @Gastos de movilidad                   & 90.00                \\
%@ & @Gastos de movilidad                   & 90.00                \\
%@ & @Otros gastos                          & 747.00               \\\midrule
%&@TOTAL@                                    & sum([0,-1]:[0,-10])        \\
%\Xhline{2pt}
%\end{spreadtab}
%  \caption{Presupuesto}\label{pp}
%\end{table}
%
%\subsection{Potencial humano}
%Personas que participaron en la realización de la investigación
%
%\begin{table}[ht]
%\centering
%\caption{Potencial humano}\label{potencialhumano}
%\begin{spreadtab}{{tabular}{ccc}}\Xhline{2pt}
%@$\text N^\circ$& @POTENCIAL HUMANO & @CANTIDAD\\\midrule
%1  & @Alumnos ($1^\circ$ artistas)           & 12 \\
%2  &@Profesores de la especialidad de escultura & 3 \\
%3  & @Autodidactas       & 2 \\
%4  & @Estadistas         & 3 \\
%5  &@Físicos             & 2 \\
%6  &@Matemáticos         & 2 \\\midrule
%&@TOTAL& sum(c2:c7)    \\
%\bottomrule
%\end{spreadtab}
%\end{table}

\renewcommand{\bibname}{REFERENCIAS BIBLIOGRÁFICAS}
\bibliographystyle{apacite}
\bibliography{une}

\vspace{-0.3cm}

\setglossarysection{section} 
\setlength{\glsdescwidth}{0.7\textwidth} 
\printglossary[nopostdot,style=long,nogroupskip,nonumberlist,type=\acronymtype,title=ACRÓNIMOS]
\vspace{-0.3cm}

\setglossarysection{section} 
\setlength{\glsdescwidth}{0.7\textwidth} 
\printglossary[nopostdot,style=long,nogroupskip,nonumberlist,type=symbols,title=SÍMBOLOS] 
\vspace{-0.3cm}

\section*{APÉNDICES}
\addcontentsline{toc}{section}{APÉNDICES}
\renewcommand{\baselinestretch}{1.0}
\pagenumbering{Roman}

\setcounter{subsection}{0}
\renewcommand{\thesubsection}{\Alph{subsection}}  
%\renewcommand{\thesubsubsection}{A.\arabic{subsection}}

\numberwithin{equation}{subsection}
\numberwithin{figure}{subsection} 
\numberwithin{table}{subsection}

\subsection{Ficha técnica de la investigación \label{ficha tecnica}}
\begin{enumerate}
	\item \textbf{Titulo }\titulo\textbf{ }
	\item \textbf{Año de realización }2019
	\item \textbf{Autor de la investigación }\autor
	\item \textbf{Asesor de la investigación }\asesor   
	\item \textbf{Colaboradores que validaron los instrumentos }\asesor, \expert y \expertt.
	\item \textbf{Institución } \lugar
	\item \textbf{Escuela profesional }\lugar
	\item \textbf{Formato }PDF.\textbf{ }
	
%	\item \textbf{Resumen }El presente trabajo de investigación tiene como objetivo 	\MakeTextLowercase{\objetivo}. Tipo de investigación básica, nivel 	de investigación experimental explicativa, de diseño cuasiexperimental 	de un grupo con pre y posprueba en series temporales equivalentes 	(alternado); se empleo el método hipotético deductivo, experimental	y estadístico descriptivo e inferencial; el lugar de estudio fue en	la \lugar; la muestra fue no probabilística e intencional compuesta	por un solo grupo experimental de 35 estudiantes de la serie 200,	matriculados en el curso de arquitectura, del semestre impar de la	escuela profesional de Ingeniería Civil UNSCH; los datos fueron recolectados	a través de la prueba escrita y la ficha de observación; la prueba 	de validez de instrumentos se realizó a través de juicio de expertos 	y la confiabilidad, a través de prueba del Coeficiente de Pearson 	y la corrección de Spearman Brow. Se verificó la no normalidad de 	los datos, mediante la prueba de  \emph{Shapiro Wilks}; se aplicó 	la prueba de Student dos muestras relacionadas para la prueba de hipótesis, 	con un nivel de confianza del 95\% y significancia del 5\% y se concluyó, 	que \MakeTextLowercase{\hipotesis}. \\Palabras Claves: \variablei, \MakeTextLowercase{\upshape \variabled, \dimddd, \dimi y \dimii}. 
	\item \textbf{Estructura de la tesis \flushbottom }
	\begin{enumerate}
		\item PLANTEAMIENTO DEL PROBLEMA 
		\begin{enumerate}
			\item Identificación y descripción del problema 
			\item Formulación del problema 
			\item Objetivos
			\item Justificación de la investigación 
		\end{enumerate}
		\item MARCO TEÓRICO 
		\begin{enumerate}
			\item Antecedentes de la investigación 
			\item Bases teóricas 
			\item Definiciones de términos básicos 
		\end{enumerate}
		\item METODOLOGÍA DE LA INVESTIGACIÓN 
		\begin{enumerate}
			\item Sistema de hipótesis 
			\item Sistema de variables 
			\item Operacionalización de variables 
			\item Aspecto metodológico 
		\end{enumerate}
		\item RESULTADOS DE LA INVESTIGACIÓN
		\begin{enumerate}
			\item Análisis e interpretación 
			\item Resultados inferenciales 
			\item Discusión de resultados 
		\end{enumerate}
		CONCLUSIONES
		
		RECOMENDACIONES
		
		REFERENCIAS BIBLIOGRÁFICAS
		
		ANEXOS
	\end{enumerate}
	\item \textbf{Antecedentes }

	\item \textbf{Problema }
	\begin{enumerate} 
		\item \textbf{Problema general.}
		
		\problema.
		\item \textbf{Problemas específicos}
		\begin{itemize}
			\item \problemae. 
			\item \problemaee. 
			\item \problemaeee. 
			\item \problemaeeee.  
		\end{itemize}
	\end{enumerate}
	\item \textbf{Objetivos }
	\begin{enumerate}
		\item \textbf{Objetivo general}
		
		\objetivo.

		\item \textbf{Objetivos específicos }
		\begin{itemize}
			\item \objetivoe. 
			\item \objetivoee. 
			\item \objetivoeee. 
			\item \objetivoeeee. 
		\end{itemize}
	\end{enumerate}
	\item \textbf{Hipótesis }
	\begin{enumerate}
		\item \textbf{Hipótesis general}
		
		\hipotesis.
		\item \textbf{Hipótesis específicas}
		\begin{itemize}
			\item \hipotesise. 
			\item \hipotesisee. 
			\item \hipotesiseee. 
			\item \hipotesiseeee. 
		\end{itemize}
	\end{enumerate}
	\item \textbf{Orientación metodológica y diseño muestral}
	\begin{itemize}
		\item \textbf{Enfoque} Cuantitativo. 
		\item \textbf{Tipo} Investigación aplicada. 
		\item \textbf{Nivel de estudio} Investigación explicativa experimental. 
		\item \textbf{Diseño} Diseño \MakeTextLowercase{\diseno} de un mismo grupo
		con pre y postprueba en series temporales equivalentes (Alternado). 
		\item \textbf{Metodología} Experimental, hipotético deductivo, estadístico
		y analítico. 
		\item \textbf{Población} Constituida por \poblacion. 
		\item \textbf{Muestra} Constituida por \muestra. 
		\item \textbf{Muestreo} No probabilístico. 
		\item \textbf{Técnica} Observación, evaluación pedagógica y experimentación. 
		\item \textbf{Instrumentos} Ficha de observación, ficha de opinión prueba
		escrita y módulos experimentales. 
		\item \textbf{Procesamiento} \textbf{de datos} Mediante los programas estadísticos R Sweave y Excel. 
	\end{itemize}
	\item \textbf{Análisis y presentación de resultados} Se procesará los datos
	recogidos en el salón de clases, es decir, se sistematizará, interpretará
	y se sintetizará para responder así a los objetivos y a la finalidad
	de la investigación, con la perspectiva, de que la presentación de
	los resultados de esta investigación sirva de reflexión, sobre el
	abordaje de la \MakeTextLowercase{\variablei}, como un factor
	fomentador del \MakeTextLowercase{\variabled}.
	\item \textbf{Conclusión }\hipotesis\textbf{.}
\end{enumerate}
\textbf{\vspace{0.5cm}
}
\begin{center}
	....................................................\\
	El investigador 
\end{center}

\clearpage

\begin{landscape}
\subsection{Matriz de consistencia\label{consistencia}}
	
	%\lugar -- TÍTULO: \titulo -- AUTOR: \autor
	
		\begin{table}[ht!]\caption{Matriz de consistencia}
			\centering \scriptsize  \renewcommand\tabcolsep{0.1cm}\renewcommand\arraystretch{1}\begin{tabular}{p{4.05cm}p{4.05cm}p{4.05cm}ccc} \hline \ce\bf PROBLEMAS &\ce\bf OBJETIVOS &\ce\bf HIPÓTESIS&\ce\bf VARIABLES&\ce\bf MARCO TEÓRICO&\bf METODOLOGÍA\Tstrut\\ \hline \ce \bf PROBLEMA GENERAL &\ce \bf OBJETIVO GENERAL &\ce \bf HIPÓTESIS GENERAL &\multirow{8}*[0cm]{\begin{minipage}[t]{2.9cm} \begin{itemize}[itemsep=-0pt,leftmargin=*,labelsep=.02cm,] \item \textbf{Variable 1}: \variablei. \begin{itemize}[itemsep=-0pt,leftmargin=*,labelsep=.02cm,topsep=-2pt] \item \dimi. \item \dimii \item \dimiii. \item \dimiiii. \end{itemize} \item \textbf{Variable 2}: \variabled. \begin{itemize}[itemsep=-0pt,leftmargin=*,labelsep=.02cm,topsep=-2pt] \item \dimd. \item \dimdd. \item \dimddd. \end{itemize} \end{itemize} \end{minipage} } &\multirow{8}*[0cm]{\begin{minipage}[t]{2.9cm} \begin{itemize}[itemsep=-0pt,leftmargin=*,labelsep=.02cm,] \item Contexto de la investigación. \item \variablei. \item \variabled. \item \dimi. \item \dimii. \item \dimiii. \item \dimiiii.\item \dimd.\item \dimdd. \item \dimddd. \end{itemize} \end{minipage} } &\multirow{22}*[0cm]{\begin{minipage}[t]{5.2cm} \begin{itemize}[itemsep=-0pt,leftmargin=*,labelsep=.02cm] \item \textbf{Enfoque de estudio:} Investigación \MakeTextLowercase{\enfoque}. \item \textbf{Tipo de estudio:} Investigación \MakeTextLowercase{\tipo}. \item \textbf{Nivel de estudio:} Investigación \MakeTextLowercase{\nivel}. \item \textbf{Diseño de estudio:} Diseño \MakeTextLowercase{\diseno} de un mismo grupo con pre y postprueba. \item \textbf{Metodología:} Experimental, hipotético deductivo, estadístico y analítico. \item \textbf{Población:} Constituida por \poblacion. \item \textbf{Muestra:} Constituida por \muestra. \item \textbf{Muestreo:} Probabilístico. \item \textbf{Técnica:} Observación, evaluación peda\-gógica y experimentación. \item \textbf{Instrumentos:} Ficha de observación, ficha de opinión prueba escrita y módulos experimentales. \item \textbf{Procesamiento de datos:} Mediante los programas estadísticos R Sweave y Excel. \end{itemize} \end{minipage} } \Tstrut\\ \cline{1-3} \problema&\objetivo&\hipotesis&&&\\ \cline{1-3} \ce\bf PROBLEMAS ESPECÍFICOS &\ce\bf OBJETIVOS ESPECÍFICOS &\ce\bf  HIPÓTESIS ESPECÍFICAS &&&\Tstrut\\\cline{1-3} \problemae&\objetivoe&\hipotesise&&&\\\cline{1-3}
				\problemaee&\objetivoee&\hipotesisee&&&\\\cline{1-3} 
				\problemaeee&\objetivoeee&\hipotesiseee&&&\\\cline{1-3} 
				\problemaeeee&\objetivoeeee&\hipotesiseeee&&&\\ \hline \end{tabular}
		\end{table}
\end{landscape}



\begin{landscape}
	\subsection{Matriz de instrumentos de la variable \MakeTextLowercase{\variablei} }
	
	\lugar -- TÍTULO: \titulo -- AUTOR: \autor
	\begin{table}[ht!]\caption{Matriz de instrumentos de la variable \MakeTextLowercase{\variablei}}
			\centering  \renewcommand\tabcolsep{0.1cm}\renewcommand\arraystretch{1}
			\scriptsize
			\begin{tabular}{cccccc}\hline			\ce\bf VARIABLE&\ce\bf DIMENSIONES &\ce\bf INDICADORES&\ce\bf ÍTEMS &\bf VALORACIÓN& \bf INSTRUMENTOS\Tstrut 			\\\hline 			
				\multirow{20}{*}{\rotatebox[origin=c]{90}{\variablei}}& 			\multirow{5}{*}{\makecell[{{M{2cm}}}]{\dimi}}  			& \multirow{2}{*}{\gb} 			&
				C1: ¿Observa y reconoce las formas modulares de los fractales en la flora?& 			\multirow{17}{*}{\makecell[{{M{2cm}}}]{Alta.\\\vspace{.2cm} 					Moderado. \\\vspace{.2cm} 					Baja.\\\vspace{.2cm} 					Muy baja.\\\vspace{.2cm} 					Nula. 			}}&\multirow{17}{*}{\makecell[{{M{2cm}}}]{Ficha de opinión\\ Prueba escrita}}\\\cline{4-4} 			&&& 
				C2: ¿Representa lo percibido, en sus creaciones artísticas?        &\\\cline{3-4} 			&& \multirow{1}{*}{\gbbb} 			&
				C3: ¿Identifica las variaciones de escala en los fractales hídricos?&\\\cline{3-4} 			&& \multirow{2}{*}{\gbbbb} 			&
				C4: ¿Observa y relaciona registros bibliográficos con los fractales atmosféricos? &\\ 			\cline{4-4}			 				&&& 
				C5: ¿Observa y relaciona registros bibliográficos con los fractales atmosféricos? &\\ 			\cline{2-4}	
				
						 			&\multirow{5}{*}{\makecell[{{M{2cm}}}]{\dimii}} 		
				& \multirow{2}{*}{\gbbbbbb}&
				C9: ¿Reconoce la recursividad de elementos fractales?\\ \cline{4-4}				&&& 
				C10: ¿Itera procesos funcionales sobre un objeto patrón?       &\\\cline{3-4} 			&&\multirow{1}{*}{\gbbbbbbb}&
				C11: ¿Reconoce transformaciones algorítmicas complejas?       &\\\cline{3-4} 			&&\multirow{2}{*}{\gbbbbbbbb}&
				C12: ¿Reconoce el movimiento Browniano?\\ \cline{4-4}					&&&
				A17: ¿Utiliza modelos de frecuencias percibidas en diversas aplicaciones?       &\\
				\cline{2-4}	
				
						 			&\multirow{5}{*}{\makecell[{{M{2cm}}}]{\dimiii}} 		
				& \multirow{2}{*}{\gbbbbbb}&
				C9: ¿Reconoce la recursividad de elementos fractales?\\ \cline{4-4}				&&& 
				C10: ¿Itera procesos funcionales sobre un objeto patrón?       &\\\cline{3-4} 			&&\multirow{1}{*}{\gbbbbbbb}&
				C11: ¿Reconoce transformaciones algorítmicas complejas?       &\\\cline{3-4} 			&&\multirow{2}{*}{\gbbbbbbbb}&
				C12: ¿Reconoce el movimiento Browniano?\\ \cline{4-4}					&&&
				A17: ¿Utiliza modelos de frecuencias percibidas en diversas aplicaciones?       &\\ 	
				\cline{2-4}	
				
						 			&\multirow{5}{*}{\makecell[{{M{2cm}}}]{\dimiiii}} 		
				& \multirow{2}{*}{\gbbbbbb}&
				C9: ¿Reconoce la recursividad de elementos fractales?\\ \cline{4-4}				&&& 
				C10: ¿Itera procesos funcionales sobre un objeto patrón?       &\\\cline{3-4} 			&&\multirow{1}{*}{\gbbbbbbb}&
				C11: ¿Reconoce transformaciones algorítmicas complejas?       &\\\cline{3-4} 			&&\multirow{2}{*}{\gbbbbbbbb}&
				C12: ¿Reconoce el movimiento Browniano?\\ \cline{4-4}					&&&
				A17: ¿Utiliza modelos de frecuencias percibidas en diversas aplicaciones?       &\\\hline 	 	
			\end{tabular} 
		\end{table}
\end{landscape}



\begin{landscape}
	\subsection{Matriz de instrumentos de la variable \MakeTextLowercase{\variabled} \label{1www}}
	
	\lugar -- TÍTULO: \titulo -- AUTOR: \autor
	\begin{table}[ht!]\caption{Matriz de instrumentos de la variable \MakeTextLowercase{\variabled}}
			\centering \scriptsize\renewcommand\tabcolsep{0.1cm}\renewcommand\arraystretch{1}
			\begin{tabular}{cccccc} 			\hline\ce\bf VARIABLE&\ce\bf DIMENSIONES &\ce\bf INDICADORES&\ce\bf ÍTEMS &\ce\bf VALORACIÓN& \bf INSTRUMENTOS \Tstrut 			\\
				\hline 		
				\multirow{20}{*}{\rotatebox[origin=c]{90}{\variabled}}& 			
				\multirow{5}{*}{\dimd} 			& 
				\multirow{2}{*}{\fb} 			&
				P1: ¿Utiliza variedad de colores?& 			
				\multirow{20}{*}{\makecell{Excelente\\\vspace{.2cm}Bueno\\\vspace{.2cm}Regular\\\vspace{.2cm}Malo\\\vspace{.2cm}Deficiente}}&
				\multirow{20}{*}{\rotatebox[origin=c]{90}{Ficha de observación}}\\\cline{4-4}	
				&&& 
				P2: ¿Plasma degradación entre los colores? &\\\cline{3-4} 			&&
				\multirow{2}{*}{\fbb} 			&
				P3: ¿Utiliza formas y superficies regulares?\\\cline{4-4}	
				&&& 
				P4: ¿Utiliza formas irregulares someramente?      &\\\cline{3-4} 	&&
				\multirow{1}{*}{\fbbb} 			&
				P5: ¿Manifiesta degradación de texturas? \\\cline{2-4} 			&
				\multirow{5}{*}{\dimdd} 			& 
				\multirow{2}{*}{\fbbbb} 			&
				P6: ¿Distribuye adecuadamente los tonos?\\\cline{4-4}	
				&&& 
				P7: ¿Asocia adecuadamente los tonos?       &\\\cline{3-4} 			&&
				\multirow{1}{*}{\fbbbbb} 			&
				P8: ¿Asocia proporcionalmente los volúmenes?\\\cline{3-4} 		&&
				\multirow{2}{*}{\fbbbbbb} 			&
				P9: ¿Conjuga texturas diversas?\\\cline{4-4}	
				&&& 
				P10: ¿Homogeneiza texturas secuencialmente?      &\\\cline{2-4} 			&
				\multirow{7}{*}{\dimddd} 			& 
				\multirow{2}{*}{\fbbbbbbb} 			&
				P11: ¿Representa elementos con dimensión creciente?\\ \cline{4-4}	
				&&&
				P12: ¿Manifiesta elementos plásticos proporcionalmente?     &\\\cline{3-4}&&
				\multirow{1}{*}{\fbbbbbbbb} 			&
				P13: ¿Representa los elementos plásticos de manera decreciente?\\\cline{3-4}&&
				\multirow{2}{*}{\fbbbbbbbb} 			&
				P14: ¿Conjuga los elementos plásticos?\\\cline{4-4}	
				&&& p15: ¿Distribuye elementos, constantes ligeramente?       &\\\cline{3-4} 			&&
				\multirow{2}{*}{\fbbbbbbbbb} 			&
				P16: ¿Yuxtapone elementos plásticos mixtos?\\\cline{4-4}	
				&&& P17: ¿Yuxtapone los ritmos en el conjunto?      &\\	\hline	\end{tabular}
		\end{table}
\end{landscape}

	\begin{enumerate}
	\item DATOS GENERALES
	\begin{itemize}
		\item Apellidos y nombres del experto: \dotfill
		\item Cargo e institución donde labora: \dotfill
		\item Nombre de los instrumentos: Ficha de observación de la  variable indendiente (INST1) y Ficha de observación de la  variable dendiente (INST2)  
		\item Titulo de la tesis: \titulo
		\item Autor de los instrumentos: \autor
	\end{itemize} 
	
	\item CRITERIOS DE VALIDACIÓN (Acuerdo: A, Descacuerdo: D)
	
\begin{center}
		\begin{tabular}{|c|p{6.8cm}|c|c|c|c|}
		\hline
		\multirow{2}*[-.01cm]{INDICADORES}	&\multirow{2}*[-.01cm]{CRITERIOS DE VALIDACIÓN} & \multicolumn{2}{c|}{INST1} & \multicolumn{2}{c|}{INST2}   \\
		\cline{3-6}
		&& A & D & A & D \\
		\hline
Claridad	& ¿Está formulado con lenguaje claro, apropiado y cencillo? &  &  &  &    \\
		\hline
		Objetividad	& ¿Las preguntas realmente recojen datos de las variables y los indicadores? &  &  &  &    \\
		\hline
		Actualización	& ¿El instrumento es adecuado para el tipo de variable en estudio? &  &  &  &    \\
		\hline
		Organización	& ¿La presentacion formal (tipo, tamaño, letra, etc.) del instrumento es apropiado? &  &  &  &    \\
		\hline
		Suficiencia	& ¿Las preguntas son suficientes para recoger datos de todos los indicadores? &  &  &  &    \\
		\hline
		Intencionalidad	& ¿Los items o preguntas responden al problema y objetivos del investigación? &  &  &  &      \\
		\hline
		Consistencia	& ¿Los items o preguntas tienen sustento teórico y científico? &  &  &  &      \\
		\hline
		Coherencia	& ¿Los items o preguntas tienen son comprensibles y están bien redactados? &  &  &  &    \\
		\hline
%		Metodología	& ¿Esta formulado con lenguaje claro, apropiado y cencillo? &  &  &  &    \\
%		\hline
%		Pertinencia	& ¿Esta formulado con lenguaje claro, apropiado y cencillo? &  &  &  &    \\
%		\hline
	\end{tabular}
\end{center}
	
	\item OPINION DE APLICABILIDAD
	
	---------------------------------------------------------------------------------------------------------------------------------------------------------------- 
\end{enumerate}

\begin{center}
	
	---------------------------\\
	Firma del experto\\
	Telefono: .....................\\
	Fecha: ......./......./.....	.......
\end{center}


\end{document} 